% !TeX root = ../sustechthesis-example.tex


\chapter[RTMQ测控系统应用:离子阱频率稳定、激光功率稳定和激光拍频稳定]{RTMQ量子测控系统应用:离子阱频率稳定、激光功率稳定和激光拍频稳定\label{section:implementation}}
RTMQ量子测控系统在有关比特控制和量子模拟策略等的相关内容的量子实验上的应用可以在我们实验室过去发表的文章\cite[]{Zhang_Wang_Wang_Zhang_Wu_Jie_Lu_2022}中参见,这里不过多阐述。本章将通过介绍使用RTMQ量子测控系统构建离子量子计算系统中几个重要子系统来进一步展示该系统的集成性和易用性。值得注意的是,RTMQ量子测控系统不仅可以完成常规的量子物理实验实时控制的需求,同时还可以不受影响地执行一些量子计算系统的实验背景环境维护,比如离子的阱频率稳定、激光功率稳定和激光拍频稳定等等。这些实验背景环境原来常常需要一大堆模拟PID控制器、滤波器、放大器等等来进行建构和实现,RTMQ系统凭借其数字化的优势而可以及大地简化整个实验仪器的使用并且使实验过程有着更强的可控性和更好的自动化水平。

% ============================================================================
% ============================================================================
% =======================     离子阱频率稳定    ===============================
% ============================================================================
% ============================================================================
\section[基于RTMQ的离子阱频率稳定]{基于RTMQ的离子阱频率稳定\label{section:trap_frequency_stablization}}

带电粒子通常由射频(RF)电势控制,其场梯度提供时间平均力,这些力构成了四极质量滤波器、离子质谱仪和射频(Paul)离子阱等应用的基础\cite[]{Dehmelt_1990, Paul_1990}。这些射频电势,通常在 1 kHz 到 100 MHz 的频率处数百或数千个伏特,在真空中驱动高阻抗负载,通常由射频放大器和谐振升压变换器(例如四分之一波或螺旋谐振器)产生\cite[]{Siverns_Simkins_Weidt_Hensinger_2012}。这种电路容易受到放大器增益、变压器的机械振动和系统中的温度漂移的波动的影响。离子阱对这些波动特别敏感,因为射频电势决定了被捕获离子的谐波振荡频率。稳定的阱频率在从量子信息处理\cite[]{Blatt_Wineland_2008, Monroe_Kim_2013}和量子模拟\cite[]{Richerme_Gong_Lee_Senko_Smith_Foss_Feig_Michalakis_Gorshkov_Monroe_2014, Jurcevic_Lanyon_Hauke_Hempel_Zoller_Blatt_Roos_2014}到原子运动的量子态的制备\cite[]{Leibfried_Blatt_Monroe_Wineland_2003}、原子干涉测量\cite[]{Johnson_Neyenhuis_Mizrahi_Wong_Campos_Monroe_2015}、和量子有限计量\cite[]{Chou_Hume_Koelemeij_Wineland_Rosenband_2010}等方面至关重要。


理论上离子阱中影响阱频率的各种因素
在离子阱系统中,阱频率的表达式为:
\begin{align}
    \omega=e\mu V_0/\sqrt{2}m\Omega R^2
\end{align}

这些参数分别是:
\begin{itemize}
    \item $e$: 离子电荷量;
    \item $\mu$: 几何效率因子;
    \item $m$: 粒子质量;
    \item $R$: 电极间距;
    \item $\Omega$: 输入微波信号的频率;
    \item $V_0$: 输入信号的电压;
\end{itemize}


上面的各个参数中$e$,$\mu$,$m$是常数,R在阱几何形状确定的情况下也是常数。因此,关键的参数在于与射频相关的 $\Omega$ 和 $V_0$ 。这其中,当今射频生成器件自身的频率和幅度稳定性是相当高的,可能产生抖动因素的实际上主要是经过谐振腔后的输出幅度。我们使用的螺线管谐振腔$Q$值较高,只要谐振腔的中心透过频率稍微发生偏移,输出幅度就可能发生较大的抖动。因此阱频稳定主要是通过稳定谐振腔输出的射频幅度实现的。

% 物理上离子阱中影响阱频率的各种因素



\subsection[离子阱频率稳定原理]{离子阱频率稳定原理}
受环境振动和温度变化影响,谐振腔的几何形状,主要是螺旋线的长度和腔体长度会发生变化,进而导致谐振腔的中心频率发生偏移。从谐振腔的S参数图\ref{fig:helical_compares}中可以看出螺线管谐振腔的Q值很大,透过峰较尖锐,即中心频率附近信号透过衰减较大。因此在输入信号频率抖动极小的情况下,受谐振腔中心频率偏移影响,信号的透过幅度将会发生较大的变化。
K. G. Johnson等人\cite[]{Johnson_Wong_Campos_Restelli_Landsman_Neyenhuis_Mizrahi_Monroe_2016}给出了一种采用模拟系统实现的离子阱频率稳定方案,通过采样和整流施加在Paul阱电极上的高压射频信号主动地对离子阱频率进行了稳定。该方案的简化版采用了鉴频器、射频功率放大器、模拟PID控制器、电容分压器、混频器、本地振荡源等器件。采用模拟方案涉及到的器件较多、灵活性差、占用实验空间大,因此我们采用基于RTMQ数字系统的稳定方式来优化上述问题。

基于RTMQ系统的离子阱频率稳定原理图如图\ref{fig:trap_frequency_lock}所示。信号通过PID对DDS进行幅度调制产生,随后经过射频功率放大器进入谐振腔耦合输入端。通过调节谐振腔耦合输入端对谐振腔进行阻抗匹配耦合,使得输出功率最大化。随后,在输出端焊接一个射频分压电路,按照100:1的分压比采集射频电压作为检测信号送入鉴频器。接着鉴频器的输出接入RTMQ板卡的16位AD转换芯片,进而转换成数字信号作为PID输入,完成整个控制回路闭环。其中,分压电路图如图\ref{fig:trap_frequency_lock}中小图所示,$C_1=C_2\approx10$uF用于同步两个输出端的射频相位;$C_3\approx0.5$pF,$C_4\approx50$pF构成理论上为$100:1$射频分压电路(实际上受到焊接电容影响测得的分压比约为$53:1$)。

\begin{figure}
    \centering
    \caption[数字PID离子阱频率稳定原理图]{基于RTMQ系统的数字PID离子阱频率稳定原理图。DDS:数字频率发生器;PID:比例微分积分控制器;ADC:模拟数字转换器;$C_1,C_2$:相位同步电容;$C_3,C_4$:射频分压电容。\label{fig:trap_frequency_lock}}
    \includegraphics[width=1.0\linewidth]{trap_frequency_lock}
\end{figure}


\subsection[基于RTMQ的离子阱频率稳定系统搭建及结果]{基于RTMQ的离子阱频率稳定系统搭建及结果}

由于系统中需要使用放大器,而放大器对不同频率的信号放大效果有所区别。过大的放大信号可能会损坏谐振腔的分压板,因此这里先用DDS的输出和放大器预先对放大器进行标定,结果如图\ref{fig:helical_lock_amplifier}所示。从结果图中的拟合直线截距上可以看出,在测试范围内输入输出功率关联性很好,信号功率放大器可以对信号放大约32.83dBm。

\begin{figure}
    \centering
    \caption[DDS输出功率和放大器输出功率]{DDS输出功率和放大器输出功率。横坐标为DDS输出功率(单位dBm),纵坐标为测量得到的放大后的射频功率(单位dBm)。蓝色实心点是测得的数据点,数据线性拟合结果为:$y=0.9644x+32.83$,系数$a=0.9644$接近1表示两者的相关性很好,截距$b=32.83$表示功率放大器的放大数约为32.83dBm。\label{fig:helical_lock_amplifier}}
    \includegraphics[width=1.0\linewidth]{helical_lock_amplifier}
\end{figure}

阱频稳定模块实验测试图如附录图\ref{fig:trap_frequency_lock_real}所示。

% \begin{figure}
%     \centering
%     \caption[阱频锁定模块实验测试图]{阱频锁定模块实验测试图\label{fig:trap_frequency_lock_real}}
%     \includegraphics[width=1.0\linewidth]{trap_frequency_lock_real}
% \end{figure}

% \subsection[离子阱频率稳定系统结果]{离子阱频率稳定系统结果}

谐振腔输出稳定效果测量结果如图\ref{fig:helical_lock_measure}所示,此时数字PID参数设置为:$k_p=2^4,k_i=1,k_d=0$。l1红色测量数据是在未进行稳定时测的的谐振腔输出幅度与输入功率设定值的关系,l2绿色和l3蓝色测量数据分别为将数字PID控制器目标值设定为-638和-909时的测量结果。在未稳定的情况下,谐振腔的输出幅度会跟随外部输入信号的幅度变化而变化,呈现出一定的比率关系。如图中红色测量数据,其斜率约为244.55;经过稳定后,在大范围内谐振腔的输出幅度仍然会跟随外部输入信号的幅度变化而变化,不过其呈现的比例关系会受到稳定抑制作用而减小,如图中绿色和蓝色测量数据,其斜率在155附近。从中可见,添加数字PID控制器对谐振腔的输出幅度会有明显的稳定抑制作用。

\begin{figure}
    \centering
    \caption[谐振腔输出稳定效果测量结果]{谐振腔输出稳定效果测量结果。数字PID参数设置为:$k_p=2^4,k_i=1,k_d=0$。横坐标为DDS的设定输出功率数字值(非实际功率值),纵坐标我板卡上ADC测得的输出电压数字值(非实际功率值)。该图个数据点测量的是PID参考功率数值一定的情况下,改变输入的射频功率大小,看实际输出功率大小的变化。l1红色线性拟合线:$y=244.55x-1420.9$;l2绿色线性拟合线:$y=154.29x-1118.6$;l3蓝色线性拟合线:$y=157.00x-1279.4$。整体上斜率越小表明控制器对功率大小的变化有抑制作用。\label{fig:helical_lock_measure}}
    \includegraphics[width=1.0\linewidth]{helical_lock_measure}
\end{figure}










% ============================================================================
% ============================================================================
% =======================      激光功率稳定     ===============================
% ============================================================================
% ============================================================================
\newpage
\section[基于RTMQ的激光功率稳定]{基于RTMQ的激光功率稳定\label{section:laser_power_locking}}
% \subsection[激光功率稳定]{激光功率稳定}
第\ref{section:quantum_computation}章\ref{section:yb_state_manipulation}节介绍了离子比特的两种操控方式——基于微波和基于激光。由于基于微波的操控在多离子比特情况下寻址困难,实际上对于多比特大规模量子计算的实现通常采用激光方案进行离子比特控制。因此,激光在基于离子的量子计算系统中扮演着十分重要的作用,而激光的功率、频率、相位等重要参数备受重视。其中最基本最重要的是激光的功率稳定,因为激光功率的波动会引入噪声从而影响离子阱中囚禁离子的量子态和量子比特门保真度\cite[]{Blums_Scarabel_Shimizu_Ghadimi_Connell_Händel_Norton_Bridge_Kielpinski_Lobino_et_al_2020}。具体来说,激光功率稳定有以下几个作用:
\begin{enumerate}
    \item 提高量子比特的质量:激光功率稳定可以减少离子阱中囚禁离子的运动和量子态的噪声,从而提高量子比特的质量和精度;
    \item 延长量子比特的寿命:激光功率稳定可以减少离子阱中囚禁离子的能量损失,从而延长量子比特的寿命;
    \item 提高量子计算的成功率:激光功率稳定可以减少离子阱中囚禁离子的运动和量子态的噪声,从而提高量子计算的成功率;
    \item 实现精确的量子操作:激光功率稳定可以实现精确的量子操作,例如囚禁离子的冷却、囚禁离子的量子门操作等;
\end{enumerate}

因此,在离子阱量子计算中,保持激光功率稳定是非常重要的,可以提高量子比特的质量、延长量子比特的寿命、提高量子计算的成功率和实现精确的量子操作。实验中现有激光功率稳定的系统多是采用模拟PID控制器来进行系统搭建的。使用模拟PID的缺点在于器件成本高、灵活性不好、稳定性差、占用实验台空间大等缺点,本部分将介绍的方案采用RTMQ系统测控板(板上芯片集成了高速数字PID模块)对经典的模拟激光功率稳定系统进行数字化优化,使得系统的稳定性、灵活性、集成度等方面有了大幅度的提高。



\subsection[激光功率外部稳定原理]{激光功率外部稳定原理}
激光功率稳定与多种因素有关,通常来说有温度控制、电流控制、光学反馈控制、机械稳定性等,这些因素基本都针对激光器本身进行稳定。除了对激光器本身进行稳定外,还可以激光输出的后续光路中设计相关的控制系统对激光进行稳定。这种方式独立于激光器吱声的出光稳定,可以在激光出光稳定的基础上进一步提高激光功率的稳定性。下面介绍这种激光功率外部稳定原理和实现。

\begin{figure}
    \centering
    \caption[激光功率外部稳定]{激光功率外部稳定。HWP:半波片,PBS:偏振片,PS(B):光功率分数器,AOM:声光调制器,PD:光功率探测器,ADC:数模转换器、PID:比例微分积分控制器,DDS:数字频率生成器、PA:功率放大器、PS(RF):微波功率分数器。\label{fig:laser_stabilization0}}
    \includegraphics[width=1.0\linewidth]{laser_stabilization0}
\end{figure}

整个激光功率外部稳定系统主要包括如图\ref{fig:laser_stabilization0}几部分:激光器、透镜组、半波片HWP、光偏振分束镜PBS、光功率分配镜PS(B)、声光调制器AOM、光分束镜PS(B)、光子探测器PD、ADC芯片、数字PID、直接数字信号合成器DDS、射频功放PA、射频功率分配器PS(RF)。

整个过程如下,首先激光器产生激光经过透镜和反光镜组将激光变换到要求的光斑大小和入射位置;接着使用半波片(HWP)将光转化为垂直和水平偏振成分;接着使用偏振分数器将偏振光提纯为匹配AOM偏振需求的光;随后使用光功率分配器分出一部分激光用于构成反馈回路,另一部分光经过控制回路控制的AOM调制后成为可用的功率稳定激光;激光经过AOM后会产生若干级衍射光,起哄主要能量集中在0级和1级衍射光上,我们可以选择其中一路来进行探测和稳定(这里选择1级光);将一级光经过光功率分束镜一部分作为测试监测,另一部分用于反馈控制回路中。控制器采用基于RTMQ板卡数字系统实现,主要器件为ADC、数字PID、DDS,除此之外还需要功率放大器(PA)和射频功率分配器(可选,如果只有一路则不用,实际使用中可以分出多路来调控多了AOM)。通过调节射频功率的大小可以控制AOM1级光和0级光的功率分布,在合理范围内,功率越大1激光分得的功率越高。数字PID根据1级衍射光的功率变化控制DDS输出不同功率的射频信号,从而实现对1级衍射激光的稳定。

对于整个控制回路,一般来说只需要分出两路如AOM0和AOM1,一路AOM0用来做功率稳定的反馈回路,另一路AOM1用来做工作光,然后可以拓展地在另一路后面添加光功率分配镜来获得更多路的稳定光,如附录图\ref{fig:laser_stabilization}所示。
% \section[激光功率稳定系统搭建]{激光功率稳定系统搭建}
激光功率稳定系统实验测试如图\ref{fig:laser_stabilization_real}所示。

\begin{figure}
    \centering
    \caption[激光功率锁定系统测试图]{激光功率锁定系统测试图\label{fig:laser_stabilization_real}}
    \includegraphics[width=1.0\linewidth]{laser_stabilization_real}
\end{figure}


\subsection[基于RTMQ的激光功率稳定系统搭建及结果]{基于RTMQ的激光功率稳定系统搭建及结果}
% D:\Database\Files\2021-2024-研究生事件文件\2022-2023\研究\2023年4月13日-光功率稳定\2023年4月27日_光功率计测试数据

数据的测量采用如图所示的RIGOL示波器进行,在50kHz的采样率下采集约3.5s的数据。激光功率外部稳定本底、稳定前、稳定后原始数据如图\ref{fig:laser_stabilization_data}所示,从图中看可以看出PD在无激光输入的暗态下本底噪声较小。除此之外,可以看到经过稳定后的激光功率小于稳定前的激光功率,这其中的主要原因是稳定过程中AOM动态分离了部分激光功率。为了能够有效地对比稳定前后的效果,我们用各组数据的均值对功率测试的结果进行归一化处理$P_{norm}=P_{orignal}/mean(P_{orignal})$,结果如图\ref{fig:laser_stabilization_data0}所示。由于本底噪声过于接近0,因此本底噪声的归一化结果呈现发散状态,在这里不具有参考价值。从图\ref{fig:laser_stabilization_data0}中可以明显看出经过数字PID稳定系统后的功率更加稳定了。

\begin{figure}
    \centering
    \caption[激光功率外部稳定原始数据]{激光功率外部稳定本底、稳定前、稳定后原始数据。横坐标:时间(单位um),纵坐标:功率(单位uW)。\label{fig:laser_stabilization_data}}
    \includegraphics[width=0.8\linewidth]{laser_stabilization_data}
\end{figure}

\begin{figure}
    \centering
    \caption[激光功率外部稳定归一化后数据]{激光功率外部稳定归一化后本底、稳定前、稳定后数据。横坐标:时间(单位um),纵坐标:功率(单位uW)。\label{fig:laser_stabilization_data0}}
    \includegraphics[width=0.8\linewidth]{laser_stabilization_data0}
\end{figure}

一般来说可以通过计算两组数据归一化后的方差之比可以粗略地获得稳定效果,方差越小稳定性越高。经计算稳定前后激光功率数据的方差比为$\frac{\sigma_{unstable}}{\sigma_{stabled}}\approx 9.17$。整体上经过激光功率稳定系统后,稳定前好了约9倍。为了能够更加直观地评估稳定的效果,我们将归一化后的所有数据点绘制出统计柱状图(由于本底的归一化不具有参考价值因此这里省略),结果如图\ref{fig:laser_stabilization_data1}所示。
\begin{figure}
    \centering
    \caption[激光功率外部稳定柱状图对比数据]{激光功率外部稳定归一化后本底、稳定前、稳定后数据柱状图。横坐标:归一化后的激光功率数值(单位1),纵坐标:频数(单位1)。\label{fig:laser_stabilization_data1}}
    \includegraphics[width=0.8\linewidth]{laser_stabilization_data1}
\end{figure}

为了能更好了解稳定效果的长时间表现情况,我们重新以3Hz的采样率采集了一组长期数据,并绘制出了归一化后稳定前后数据的阿兰方差,如图\ref{fig:laser_stabilization_data2}所示。从结果对比来看,稳定后的激光功率数据的阿兰方差在各个滑动窗口都低于未稳定时的。表明这里的数字稳定系统在较长时间的各个滑动窗口稳定性都好于未稳定时的。
\begin{figure}
    \centering
    \caption[激光功率外部稳定阿兰方差对比数据]{激光功率外部稳定归一化后本底、稳定前、稳定后数据阿兰方差。数据采样频率:3 Hz,横坐标:窗口长度(单位s),纵坐标:阿兰方差(单位1)。蓝色实线为没有稳定情况下的功率阿兰方差,橙色实线是稳定后的功率阿兰方差。\label{fig:laser_stabilization_data2}}
    \includegraphics[width=0.8\linewidth]{laser_stabilization_data2}
\end{figure}










% ============================================================================
% ============================================================================
% =======================      激光拍频稳定     ===============================
% ============================================================================
% ============================================================================
\newpage
\section[基于RTMQ的脉冲激光拍频稳定]{基于RTMQ的脉冲激光拍频稳定\label{section:pulsed_laser_locking}}

% \textcolor{red}{这部分叙述主要参考文献\cite[]{Islam_Campbell_Choi_Clark_Conover_Debnath_Edwards_Fields_Hayes_Hucul_et_al_2014}}
量子信息通常会被编码在量子系统的不同能级结构中,比如第\ref{section:yb_computation}节中介绍的镱离子的内能级。这些能级能量差一般在微波或者光学波段,其量子比特状态通常可以用外部场操纵,例如微波或光场。在能级与外部场的耦合下,可以驱动量子状态发生改变,也即实现\emph{量子比特操控}。锁模(脉冲)激光器是可以实现此目的的通用仪器,其宽带光谱同时具有射频(RF)和微波结构。这种激光器已被用于控制原子\cite[]{Hayes_Matsukevich_Maunz_Hucul_Quraishi_Olmschenk_Campbell_Mizrahi_Senko_Monroe_2010}、分子\cite[]{Peer_Shapiro_Stowe_Shapiro_Ye_2007}和固态量子系统\cite[]{Greve_Press_McMahon_Yamamoto_2013}。本节中,我将介绍一种简单的技术来稳定这些光源的拍频,以便操作和控制通用量子比特系统\cite[]{ladd2010quantum}。

\subsection[脉冲激光拍频稳定原理]{脉冲激光拍频稳定原理}

% \subsubsection[量子信息的调控频率]{量子信息的调控频率}




\subsubsection[锁模(脉冲)激光器]{锁模(脉冲)激光器}
如第\ref{section:pulsed_laser_ion_operation}节中所述,锁模(脉冲)激光器可用于产生宽带光学频率梳,总带宽从$10$GHz到$100 $THz,梳齿由激光的重复速率$\omega_{rep}$间隔,通常在$0.1-1 $GHz范围内。量子比特的能级劈裂会匹配到重复频率的整数倍数附近,为了弥合与特定量子比特的频率差距,需要使用额外的光调制器完成微调\cite[]{Hayes_Matsukevich_Maunz_Hucul_Quraishi_Olmschenk_Campbell_Mizrahi_Senko_Monroe_2010}。为了在量子系统中保持长期相干性,与激光源相关的光学频率差必须稳定\cite[]{Stick_Hensinger_Olmschenk_Madsen_Schwab_Monroe_2006}。

产生稳定频率差的一种方便方法是将光源的光分成两条路径,将其中一束光进行微调,然后让它们在离子比特处交汇。这种方式可以避免激光共模波动引起的频率抖动问题\cite[]{Thomas_Hemmer_Ezekiel_Leiby_Picard_Willis_2002}。
一般来说,对于跃迁频率$\nu_{ab}\leq 1$GHz的情况,可以使用声光调制器(Acousto-Optic Modulator, AOM)产生相应的频率;对于高达$\nu_{ab} \sim 10 $GHz的较大跃迁频率,可以使用电光调制器(Electro-Optic Modulator, EOM),尽管由于频率调制边带会受到相位偏移模式影响被抑制\cite[]{Lee_Blinov_Brickman_Deslauriers_Madsen_Miller_Moehring_Stick_Monroe_2003}。锁模(脉冲)激光器可以产生不受此相位问题影响的调幅(AM)边带,成为生成调控边带的极佳方案。锁模(脉冲)激光器的带宽通常足够大,可以解决频率$\nu_{ab} < 100 $THz的能级劈裂。与AOM的适度频移相结合,可用于在该带宽内达到任意频率。


\subsubsection[脉冲激光的拍频探测原理]{脉冲激光的拍频探测原理}
对光频率的拍频探测是一个十分重要的环节,本质上对光频率的拍频探测就是对光能量的探测。对激光频率的拍频探测通常采用光电二极管组成的光探测器(Photon Detector, PD)。它的探测结果受到光电二极管的响应速度影响,比如响应速度小于1GHz的PD会过滤掉激光中超过1GHz的高频成分。我们的目标频率是光拍频结果中与离子比特能级共振的频率,即12.56GHz附近的频率。因此系统的搭建需要使用超快光探测器。接下来我将具体介绍对脉冲激光的拍频探测原理。

脉冲激光的一般时域表示,考虑脉冲激光是幅度调制的连续激光:
\begin{align}
    E(t)=E_0 (t) e^{-i\omega_0 t}
\end{align}

从傅里叶分析角度表述:
\begin{align}
    E(t)=\left(\sum_{n=-\infty}^{\infty}\left[E_n e^{-in\omega_{rep}t} e^{-i\omega_0 t} \right]\right)\cdot e^{ikx}\cdot e^{i\phi}
\end{align}
其中$E_n$是频率为$n\omega_{rep}$的光频成分对应的电场大小,$k$是脉冲光的波矢,$\phi$是脉冲光的初始相位。

用超快光电探测器(Ultra-fast Photon Detector, UPD)探测一路脉冲光时得到光功率:
\begin{footnotesize}
\begin{align}
    P(t)=&|E(t)|^2=\left|\left(\sum_{n=-\infty}^{\infty}\left[E_n e^{-in\omega_{rep}t} e^{-i\omega_0 t} \right]\right)\cdot e^{ikx}\cdot e^{i\phi}\right|^2\\
    =&\left\{\left(\sum_{n=-\infty}^{\infty}\left[E_n e^{-in\omega_{rep}t} e^{-iω_0 t} \right]\right)\cdot e^{ikx}\cdot e^{i\phi}\right\} \cdot\left\{\left(\sum_{m=-\infty}^{\infty}\left[E^*_m e^{im\omega_{rep}t} e^{i\omega_0 t} \right]\right)\cdot e^{-ikx}\cdot e^{-i\phi}\right\}\\
    =&\sum_{n,m=-\infty}^{\infty}E_n e^{-in\omega_{rep} t} e^{-i\omega_0 t}\cdot E_m^* e^{im\omega_{rep} t} e^{i\omega_0 t}\\
    =&\sum_{n,m=-\infty}^{infty}E_n E_m^* e^{-i(n-m) \omega_{rep} t}\label{eq:pt_res1}
\end{align}    
\end{footnotesize}

针对公式\eqref{eq:pt_res1}分别考虑$n,m$的几类可能情况,原方程可化为:
\begin{footnotesize}
\begin{align}
    P(t)=&\sum_{n=m=-\infty}^{\infty}E_n E_m^*
    +\sum_{n>m=-\infty}^{\infty}E_n E_m^*e^{-i(n-m)\omega_{rep}t}+\sum_{n<m=-\infty}^{\infty}E_n E_m^*e^{-i(n-m)\omega_{rep}t}\\
    =&\sum_{n=m=-\infty}^{\infty}E_n E_m^*
    +\sum_{n>m=-\infty}^{\infty}E_n E_m^*e^{-i(n-m)\omega_{rep}t}+\sum_{m<m=-\infty}^{\infty}E_m E_n^*e^{-i(m-n)\omega_{rep}t}\\
    =&\sum_{n=m=-\infty}^{\infty}E_n E_m^*+\sum_{n>m=-\infty}^{\infty}E_n E_m^*e^{-i(n-m)\omega_{rep}t}+E_mE_n^* e^{i(n-m)\omega_{rep}t}
\end{align}
\end{footnotesize}

考虑电场强度为实数的情况下,可得:
\begin{align}
    P(t)=&\sum_{n=m=-\infty}^{\infty}E_n^2 + \sum_{n>m=-\infty}^{\infty}E_n E_m\left(e^{-i(n-m)\omega_{rep}t}+e^{i(n-m)\omega_{rep}t}\right)\\
    =&\sum_{n=m=-\infty}^{\infty}E_n^2 + \sum_{n>m=-\infty}^{\infty}E_n E_m\left(2\cos\left((n-m)\omega_{rep}t\right)\right)\\
    =&\sum_{n=m=-\infty}^{\infty}E_n^2 + \sum_{n>m=-\infty}^{\infty}2E_n E_m \cos\left((n-m)\omega_{rep}t\right)\label{eq:pt_res2}
\end{align}

这其中第一项为与时间无关项,反映到探测结果中就是UPD的直流成分;第二项是一些列的三角函数波,它们幅度相等,并具有$\omega_{rep}$的频率间隔,这便是单路脉冲光探测结果中的频率梳成分。

我们的方案是采用两路光同时作用在离子上的,下面讨论同时探测两路光得到的结果。假设两路光分别是$E_1(t),\ E_2(t)$,它们的探测结果为:
\begin{footnotesize}
\begin{align}
    P(t)=&\left|\left(\sum_{n_1=-\infty}^{\infty}E_{n_1}e^{-in_1\omega_{rep1}t}e^{-i\omega_{01}t}\right)\cdot e^{ik_1x}\cdot e^{i\phi_1} + \left(\sum_{n_2=-\infty}^{\infty}E_{n_2}e^{-in_2\omega_{rep2}t}e^{-i\omega_{02}t}\right)\cdot e^{ik_2x}\cdot e^{i\phi_2}\right|^2\\
    =&\left\{\left(\sum_{n_1=-\infty}^{\infty}E_{n_1}e^{-in_1\omega_{rep1}t}e^{-i\omega_{01}t}\right)\cdot e^{ik_1x}\cdot e^{i\phi_1} + \left(\sum_{n_2=-\infty}^{\infty}E_{n_2}e^{-in_2\omega_{rep2}t}e^{-i\omega_{02}t}\right)\cdot e^{ik_2x}\cdot e^{i\phi_2}\right\}\\
    +&\left\{\left(\sum_{m_1=-\infty}^{\infty}E_{m_1}^* e^{im_1\omega_{rep1}t}e^{i\omega_{01}t}\right)\cdot e^{-ik_1x}\cdot e^{-i\phi_1} + \left(\sum_{m_2=-\infty}^{\infty}E_{m_2}^*e^{im_2\omega_{rep2}t}e^{i\omega_{02}t}\right)\cdot e^{-ik_2x}\cdot e^{-i\phi_2}\right\}\label{eq:pt_res3}
\end{align}
\end{footnotesize}


公式\eqref{eq:pt_res3}的乘积展开有四项,分别为:
\begin{footnotesize}
\begin{align}
    P(t)=&RES1+RES2+RES3+RES4\\
    RES1=P_{1to1}(t)=&\left(\sum_{n_1=-\infty}^{\infty}E_{n_1}e^{-in_1\omega_{rep1}t}e^{-i\omega_{01}t}\right)\cdot e^{ik_1x}\cdot e^{i\phi_1} 
    \cdot \left(\sum_{m_1=-\infty}^{\infty}E_{m_1}^* e^{im_1\omega_{rep1}t}e^{i\omega_{01}t}\right)\cdot e^{-ik_1x}\cdot e^{-i\phi_1}\\
    RES2=P_{1to2}(t)=&\left(\sum_{n_1=-\infty}^{\infty}E_{n_1}e^{-in_1\omega_{rep1}t}e^{-i\omega_{01}t}\right)\cdot e^{ik_1x}\cdot e^{i\phi_1} 
    \cdot \left(\sum_{m_2=-\infty}^{\infty}E_{m_2}^*e^{im_2\omega_{rep2}t}e^{i\omega_{02}t}\right)\cdot e^{-ik_2x}\cdot e^{-i\phi_2}\\
    RES3=P_{2to1}(t)=&\left(\sum_{n_2=-\infty}^{\infty}E_{n_2}e^{-in_2\omega_{rep2}t}e^{-i\omega_{02}t}\right)\cdot e^{ik_2x}\cdot e^{i\phi_2}
    \cdot \left(\sum_{m_1=-\infty}^{\infty}E_{m_1}^* e^{im_1\omega_{rep1}t}e^{i\omega_{01}t}\right)\cdot e^{-ik_1x}\cdot e^{-i\phi_1}\\
    RES4=P_{2to2}(t)=&\left(\sum_{n_2=-\infty}^{\infty}E_{n_2}e^{-in_2\omega_{rep2}t}e^{-i\omega_{02}t}\right)\cdot e^{ik_2x}\cdot e^{i\phi_2}
    \cdot \left(\sum_{m_2=-\infty}^{\infty}E_{m_2}^*e^{im_2\omega_{rep2}t}e^{i\omega_{02}t}\right)\cdot e^{-ik_2x}\cdot e^{-i\phi_2}
\end{align}
\end{footnotesize}

其中RES1和RES4是单路脉冲光自相互作用的结果项,RES2和RES3是两路单路脉冲相互作用的结果项。RES1和RES4这种单路脉冲光自相互作用的探测结果前面公式\eqref{eq:pt_res2}已经讨论过了,结果可以直接类似写出如下:

\begin{align}
    RES1=P_{1to1}(t)=&\sum_{n_1=m_1=-\infty}^{\infty}E_{n_1}^2 + \sum_{n_1>m_1=-\infty}^{\infty}2E_{n_1} E_{m_1} \cos\left((n_1-m_1)\omega_{rep1}t\right)\label{eq:final_res1}\\
    RES4=P_{2to2}(t)=&\sum_{n_2=m_2=-\infty}^{\infty}E_{n_2}^2 + \sum_{n_2>m_2=-\infty}^{\infty}2E_{n_2} E_{m_2} \cos\left((n_2-m_2)\omega_{rep2}t\right)\label{eq:final_res4}
\end{align}


下面重点讨论RES2和RES3这类两路光相互作用的项。
\begin{footnotesize}
\begin{align}
    RES2=P_{1to2}(t)=&\left(\sum_{n_1=-\infty}^{\infty}E_{n_1}e^{-in_1\omega_{rep1}t}e^{-i\omega_{01}t}\right)\cdot e^{ik_1x}\cdot e^{i\phi_1} 
    \cdot \left(\sum_{m_2=-\infty}^{\infty}E_{m_2}^*e^{im_2\omega_{rep2}t}e^{i\omega_{02}t}\right)\cdot e^{-ik_2x}\cdot e^{-i\phi_2}\\
    =&\sum_{n_1,m_1=-\infty}^{\infty}E_{n_1}E_{m_2}^*e_1^{-i(n_1\omega_{rep1}-m_2\omega_{rep2})t}e^{-i(\omega_{01}-\omega_{02})t}e^{i(k_1-k_2)x}e^{i(\phi_1-\phi_2)}
\end{align}
\end{footnotesize}

此时考虑具体情况,这两束光为同一束脉冲光中分出的,则有$E_{n_2}=kE_{n_1}$,$\omega_{rep1}=\omega_{rep2}$,此时上面结果变为:
\begin{footnotesize}
\begin{align}
    RES2=P_{1to2}(t)=&\sum_{n_1,m_1=-\infty}^{\infty}E_{n_1}kE_{m_2}^*e_1^{-i(n_1-m_2)\omega_{rep1}t}e^{-i(\omega_{01}-\omega_{02})t}e^{i(k_1-k_2)x}e^{i(\phi_1-\phi_2)}
\end{align}
\end{footnotesize}

前面计算过电场为实数时的情况,类似的可以得到:
\begin{align}
    \sum_{n,m=-\infty}^{\infty}E_nE_m^*e^{-i(n-m)\omega_{rep}t}=\sum_{n=m=-\infty}^{\infty}E_n^2+\sum_{n>m=-\infty}^{\infty}2E_nE_m\cos((n-m)\omega_{rep}t)
\end{align}

此时,记e指数项相关的变量$\Delta\omega=\omega_{01}-\omega_{02},\ \Delta k=k_1-k_2,\ \Delta\phi=\phi_1-\phi_2$,RES2结果变为:
\begin{footnotesize}
\begin{align}
    RES2=P_{1to2}(t)=&\left\{\sum_{n_1=m_2=-\infty}^{\infty}kE_{n_1}^2+\sum_{n_1>m_2=-\infty}^{\infty}2E_{n_1}E_{m_2}(e^{-i(n_1-m_2)\omega_{rep1}t}+e^{i(n_1-m_2)\omega_{rep1}t})\right\}\\
    & \cdot e^{-i(\Delta\omega)t}e^{i(\Delta k)x}e^{i\Delta\phi}\\
    =&\left\{\sum_{n_1=m_2=-\infty}^{\infty}kE_{n_1}^2+\sum_{n_1>m_2=-\infty}^{\infty}2E_{n_1}E_{m_2}\cos((n_1-m_2)\omega_{rep}t)\right\}\notag\\
    & \cdot (\cos\Delta\omega t-i\sin\Delta\omega t)e^{i(\Delta k)x}e^{i\Delta\phi}\label{eq:final_res2}
\end{align}
\end{footnotesize}

同理,RES3的计算可以相应地得到。若记$\Delta\omega=\omega_{02}-\omega_{01},\ \Delta k=k_2-k_1,\ \Delta\phi=\phi_2-\phi_1$,则RES2最后的结果与公式\eqref{eq:final_res2}类似,结果如下:
\begin{small}
\begin{align}
    RES3=P_{2to1}(t)=&\left\{\sum_{n_1=m_2=-\infty}^{\infty}kE_{n_1}^2+\sum_{n_1>m_2=-\infty}^{\infty}2E_{n_1}E_{m_2}\cos((n_1-m_2)\omega_{rep}t)\right\}\notag\\
    & \cdot (\cos\Delta\omega t+i\sin\Delta\omega t)e^{-i(\Delta k)x}e^{-i\Delta\phi}\label{eq:final_res3}
\end{align}    
\end{small}

最后,两路来自同一脉冲束的光作用的探测结果$P(t)$可以由公式\eqref{eq:final_res1}公式\eqref{eq:final_res2}公式\eqref{eq:final_res3}公式\eqref{eq:final_res4}结果共同写出:
\begin{small}
\begin{align}
    P(t)=&\sum_{i=1}^{4}RES_i\\
    =&\sum_{n_1=m_1=-\infty}^{\infty}E_{n_1}^2 + \sum_{n_1>m_1=-\infty}^{\infty}2E_{n_1} E_{m_1} \cos\left((n_1-m_1)\omega_{rep1}t\right)\\
    &+\left\{\sum_{n_1=m_2=-\infty}^{\infty}kE_{n_1}^2+\sum_{n_1>m_2=-\infty}^{\infty}2E_{n_1}E_{m_2}\cos((n_1-m_2)\omega_{rep}t)\right\}\notag\\
    & \cdot (\cos\Delta\omega t-i\sin\Delta\omega t)e^{i(\Delta k)x}e^{i\Delta\phi}\\
    &+\left\{\sum_{n_1=m_2=-\infty}^{\infty}kE_{n_1}^2+\sum_{n_1>m_2=-\infty}^{\infty}2E_{n_1}E_{m_2}\cos((n_1-m_2)\omega_{rep}t)\right\}\notag\\
    & \cdot (\cos\Delta\omega t+i\sin\Delta\omega t)e^{-i(\Delta k)x}e^{-i\Delta\phi}\\
    &+\sum_{n_2=m_2=-\infty}^{\infty}E_{n_2}^2 + \sum_{n_2>m_2=-\infty}^{\infty}2E_{n_2} E_{m_2} \cos\left((n_2-m_2)\omega_{rep2}t\right)
\end{align}
\end{small}

经一部化简得到,两路来自同一脉冲束的光作用的探测结果$P(t)$表达式为:
% \begin{small}
\begin{align}
    P(t)=&\sum_{n_1=m_1=-\infty}^{\infty}(k^2+1)E_{n_1}^2\\
    &+\sum_{n_1>m_1=-\infty}^{\infty}2(k^2+1)E_{n_1}E_{m_1}\cos\left((n_1-m_1)\omega_{rep}t\right)\\
    &+\sum_{n_1=m_1=-\infty}^{\infty}2kE_{n_1}^2\cos\Delta\omega t\\
    &+\sum_{n_1>m_1=-\infty}^{\infty}2kE_{n_1}E_{m_1}\cos\left((n_1-m_1)\omega_{rep}t\right)cos\Delta\omega t
\end{align}
% \end{small}

这其中第一项贡献了探测结果中的直流项;第二项贡献了结果中的间隔频率为$\omega_{rep}$的频率梳齿项;第三项贡献了频率为调制频率差$\Delta\omega$的项;第四项贡献了各个频率梳齿两侧相距中心梳齿$\Delta\omega$的边带项。第四项的频率梳齿边带正是要稳定的目标频率项,里面包含了外部可调节项$\Delta\omega$,而这一项来可以是AOM。我们可以通过设计控制回路,通过AOM输出控制信号$\Delta\omega$,进而使得目标频率$(n_1-m_1)\omega_{rep}\pm \Delta\omega$稳定在量子比特编码能级频率差$\nu_{qubit}$上。



\subsubsection[基于RTMQ系统的激光拍频稳定]{基于RTMQ系统的激光拍频稳定}

\begin{figure}
    \centering
    \caption[脉冲激光拍频稳定系统框图]{脉冲激光拍频稳定系统框图。AOM:声光调制器;PD:光功率探测器;DDS:数字频率生成器;PID:比例积分微分控制器;FOL:频率低通滤波器;ADC:模拟数字转化器。\label{fig:beat_note_stabilization}}
    \includegraphics[width=1.0\linewidth]{beat_note_stabilization}
\end{figure}

如图\ref{fig:beat_note_stabilization}所示,为了调整驱动量子比特跃迁的场的频率,来自锁模(脉冲)激光器的光束被分成两条路径,在各自支路上分别受到$\nu_{M1}$和$\nu_{M2}$的频率调制,然后在离子比特上重新交汇。
两条光路尽量等长(不必须严格等长或在光学波长尺度上长期稳定),并需要微调光路长度以使得脉冲能在离子比特处交汇。脉冲激光拍频稳定系统实物如图\ref{fig:beat_note_stabilization_real}所示。

\begin{figure}
    \centering
    \caption[脉冲激光拍频锁定实验系统图]{脉冲激光拍频锁定实验系统图\label{fig:beat_note_stabilization_real}}
    \includegraphics[width=1.0\linewidth]{beat_note_stabilization_real}
\end{figure}

图\ref{fig:frequency_comb}显示了由快速光电二极管测量的光束的结果RF光谱,由每个单独光束的激光重复率$\nu_{rep}$分隔,以及经AOM调制后组合光束的$\nu_{rep}\pm |\Delta \nu_M|$分隔的附加拍频边带,其中$|\Delta \nu_M|=|\nu_{M1}-\nu_{M2}|$。
我们通过调整路径$\Delta\nu_M$的相对频移来控制这些额外边带的位置,使之与编码量子信息的能级共振。经调制后用于调控离子的目标边带为$\nu_{sb}$,可以表达为:
\begin{align}
    \nu_{sb}=n\nu_{rep}\pm|\Delta\nu_M|
\end{align}

\begin{figure}
    \centering
    \caption[脉冲激光频率梳]{在频谱分析仪中看到的脉冲激光频率梳。激光频率梳由每个单独光束的激光重复率$\nu_{rep}$分隔,以及经AOM调制后组合光束的$\nu_{rep}\pm |\Delta \nu_M|$分隔的附加拍频边带,其中$|\Delta \nu_M|=|\nu_{M1}-\nu_{M2}|$。\label{fig:frequency_comb}}
    \includegraphics[width=1.0\linewidth]{frequency_comb}
\end{figure}

其中$n$是某个整数,$\nu_{sb}$在频率梳的带宽内。注意采用拉曼跃迁的方式操控时(第\ref{section:raman_transition}节),如图\ref{fig:raman_transition1}所示,激光载波包络相位不需要稳定\cite[]{Peer_Shapiro_Stowe_Shapiro_Ye_2007},尽管也有类似的方式可以完成\cite[]{Koke_Grebing_Frei_Anderson_Assion_Steinmeyer_2010}。

\begin{figure}
    \centering
    \caption[受激拉曼跃迁示意图]{受激拉曼跃迁示意图。$\ket{e}$:上级激发态能级;$\ket{a},\ket{b}$:编码量子信息的高低能级;$\nu_{ab}$:编码量子信息的能级能隙;$\nu_{a}$:$\ket{e}$能级与$\ket{b}$能级的能隙;$\nu_a^L$:$\ket{a}$能级与虚拟能级的能隙;$\nu_b^L$:$\ket{b}$能级与虚拟能级的能隙。\label{fig:raman_transition1}}
    \includegraphics[width=0.6\linewidth]{raman_transition1}
\end{figure}

驱动量子比特需要梳齿之间的光学相干性在量子比特操作的毫秒或更快的时间尺度上保持,这对于锁模(脉冲)激光源来说很容易实现\cite[]{Hayes_Matsukevich_Maunz_Hucul_Quraishi_Olmschenk_Campbell_Mizrahi_Senko_Monroe_2010}。然而,由于激光腔长中的热应变或其它机械应变,激光重复频率会产生漂移[$\nu_{rep}=\nu_{rep}(t)$],导致施加的边带频率$\nu_{sb}$也跟着漂移,进一步地对量子比特驱动也将偏离共振。因此需要一定的措施来稳定用于操控量子比特的边带频率。

可以像密封或光纤激光器的情况一样直接将误差信号反馈给激光腔长,从而稳定重复率,不过当激光腔不容易访问时,这将十分困难。此外,这样的锁的带宽将有限,因为测量激光重复率的采集时间和使用机械换能器调制激光腔长度的延迟可能比激光腔波动的特征时间长。例如,$^{171}Yb$($12.642819 $GHz)电子基态的超精细跃迁在$ \nu_{rep} \sim 80.65$MHz处接近锁模频率激光器的第 157 梳齿。
想要将上述频率稳定在1kHz以内的话,一种经典的方法是测量$\nu_{rep}(t)$使其达到几赫兹的分辨率,而这需要的积分时间将超过1s。这种过于慢的稳定方式并不适合用来稳定频率梳齿的拍频边带。

作为替代,我们选择使用频率梳齿的$n\nu_{rep}(t)$与本地标准参考信号$\nu_{LO}$进行拍频的方式来检测频率的漂动。如图\ref{fig:beat_note_stabilization}所示,我们将锁模(脉冲)激光分出一部分与用FPD进行探测,然后经过一个12.5GHz为中心的带通滤波器并放大后,将其与在12.56GHz附近的本地振荡源拍频。两者拍频之后经过一个合适的低通或带通滤波器就可以获得一个差频信号$|n\nu_{rep}-\nu_{LO}|$(两者的绝对大小不重要,我们总是可以观测到正的差频)。接着将这个差频信号放大并与板卡的DDS输出频率进行拍频和滤波获得一个更小的差频信号,将这个信号送入板卡的AD进行模数转换,随后经过数字滤波器和数字PID的处理对DDS的频率进行调节。整个控制器是采用数字系统实现的,主要涉及模块包括16位AD转换器(Linear Technology, LTC2216)、16位滤波器(第\ref{section:digital_iir}节)、16位PID(第\ref{section:digital_pid}节)、32位DDS(Analog Device, AD9910)。






\subsection[基于RTMQ的脉冲激光拍频稳定系统搭建及结果]{基于RTMQ的脉冲激光拍频稳定系统搭建及结果}

激光拍频稳定的结果如图\ref{fig:beat_note_stabilization_signal}所示,图中展示了在12.39GHz附近的一个稳定的拍频边带。实验中可以利用此类似这样的稳定后的边带与离子进行相互作用,进而实现对离子量子比特的调控。值得一提的是,尽管系统稳定是针对某一级(例如157级)频率梳齿的边带进行稳定的,实际上整个频率梳齿的边带都会一定程度上被稳定下来。不过由于锁模(脉冲)激光重复频率的长期漂动,距离目标稳定边带越远的边带稳定效果越差。
\begin{figure}
    \centering
    \caption[脉冲激光拍频稳定12.39GHz附近稳定边带结果]{脉冲激光拍频稳定12.39GHz附近稳定边带结果\label{fig:beat_note_stabilization_signal}}
    \includegraphics[width=1.0\linewidth]{beat_note_stabilization_signal}
\end{figure}

\subsubsection[拍频稳定系统的可锁频率幅度]{拍频稳定系统的可锁频率幅度}
由于锁模(脉冲)激光重复频率的长期漂动,数字PID拍频稳定系统需要对其进行补偿来使得频率符合预期值。然而拍频稳定系统并不能应付无限大的频率漂移。\emph{可锁频率幅度}表示了锁频系统对拍频结果的长期漂移的最大可接受范围,超过此范围的长期频率漂动将无法被稳定。下面讨论关于系统的可锁频率幅度相关问题。

该系统结合了大量数字系统,它的控制器和频率输出都由数字系统实现,这及大地提高了系统的稳定性和灵活性。控制器采用了数字PID,调节频率输出采用了DDS。因此理论上只要DDS的输出可以跟得上,应该能稳定接近DDS输出的可调节频率幅度。系统中采用的DDS的频率编码为32bit,频率范围为400MHz。寄存器的数值$R_{DDS}$和实际输出的模拟频率$f_{DDS}$的换算关系为:
\begin{align}
    f_{DDS}=\frac{R_{DDS}}{2^{32}}\times 400\ \textnormal{MHz}
\end{align}

DDS数字对应的\emph{频率步长}为$f_{resolution}=\frac{400\textnormal{MHz}}{2^{32}}\approx0.093\textnormal{Hz}$。然而,DDS芯片中给出的并行频率调制位数是16位,而实际的频率分辨率是32位,这意味着没有办法并行进行全部(0-400MHz)范围内的频率调制。
实际的使用中可以设定一个中心频率$f_{c}$,然后通过16位并行调制字再对这个频率进行微调$\Delta f$。另外,这16位并行调制字可以通过移位$N_{shift}$来对应的32位频率调制字的不同位数,以获得不同的调节范围。每左移一位,并行调制字的权重乘2,最大可移位数为15。也即,16位频率调制字的调制效果$\Delta f$为:
\begin{align}
    \Delta f=f_{resolution}\times n \times 2^{N_{shift}}
\end{align}

其中$n$是16位频率调制字对应的十进制无符号数。最终,DDS的并行调制频率结果为:
\begin{align}
    f_{dds}=f_c+f_{resolution}\times n \times 2^{N_{shift}}
\end{align}

需要注意的是在每次的位移后,并行调制的调制频率步长也会增加2倍,也即频率分辨率会相应下降2倍。频率调制字位移数与调制步长的关系如图\ref{fig:beat_note_resolution_shift}所示,频率调制字位移数与调制精度两者满足关系$f_{precision}=f_{resolution}\times2^{N_{shift}}$。

\begin{figure}
    \centering
    \caption[频率调制字位移数vs调制精度
    ]{频率调制字位移数vs调制精度。频率调制字位移数与调制精度两者满足关系$f_{precision}=f_{resolution}\times2^{N_{shift}}$。
    \label{fig:beat_note_resolution_shift}}
    \includegraphics[width=1.0\linewidth]{beat_note_resolution_shift}
\end{figure}


\begin{figure}
    \centering
    \caption[脉冲激光拍频稳定可锁频率幅度测试]{脉冲激光拍频稳定可锁频率幅度测试。蓝色虚线给出了前7个数据点关于公式$f(x)=a\times e^{b\times x}+c$的拟合结果,结果为:a=21.54,95\%置信区间(4.323,38.76);b=0.7487 ,95\%置信区间(0.5666,0.9308);c=-18.03,95\%置信区间(-56.06,20)。\label{fig:beat_note_lockable_range_bits}}
    \includegraphics[width=1.0\linewidth]{beat_note_lockable_range_bits}
\end{figure}

通过计算可知,当16位并行调制字移位为0时,可以调节的频率范围仅为约6kHz。在数字PID和DDS的连接一定的情况下,为了测试整个拍频稳定系统的最大可锁频率幅度,可以通过不断移位并行调制字,来测得整个锁频系统最大的可锁频率幅度。测量结果如图\ref{fig:beat_note_lockable_range_bits}所示,理论上在移位到15之前每次移位后测的的可锁频率幅度相对前一种情况应该乘2。图中蓝色虚线给出了前7个数据点关于公式$f(x)=a\times e^{b\times x}+c$的拟合结果,结果为:a=21.54,95\%置信区间(4.323,38.76);b=0.7487 ,95\%置信区间(0.5666,0.9308);c=-18.03,95\%置信区间(-56.06,20)。

从图\ref{fig:beat_note_lockable_range_bits}中可见在移位0-6之间时基本符合可锁频率幅度岁位移数依次增大2倍这个规律,而到6之后可锁频率幅度便基本保持稳定了。从中可以得出在当前配置下,整个系统的最大可锁频率幅度约为450kHz。这个数值的限制因素主要可能有两个原因:1. 频率调制分辨率过低,导致无法稳定到目标频率上;2. 数字PID带宽影响;3. 系统中存在限制带宽的器件。联系到为了提升调制品质和减少噪声影响,我们的数字系统在PID前添加了一个数字低通滤波器,它的带宽约为450kHz。因此该数字低通滤波器应该是这个结果的主要影响因素。不过由于450kHz的可锁频率范围完全够用了,可以不用再这方面进行进一步优化。如果需要更宽的可锁频率范围,可以通过重新设计该数字低通滤波器的形状和带宽来方便地实现。






