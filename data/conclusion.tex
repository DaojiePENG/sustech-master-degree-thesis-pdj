% !TeX root = ../sustechthesis-example.tex

\begin{conclusion}

量子计算拥有着广阔的发展和应用前景。离子阱量子计算因其较高的相干时间和保真度是量子计算未来重要的发展方向之一。离子阱系统是一个综合复杂的系统,其中测控系统占有十分关键的位置。本文主要研究内容为面向离子阱量子计算的实验测控系统,介绍了与之相关的离子阱量子计算物理背景、专为量子实验设计的RTMQ测控系统、用于离子阱的螺线管谐振腔以及基于前述内容实现的若干重要量子计算子系统。

第\ref{section:quantum_computation}章介绍了量子测控系统的重要研究背景基础。以离子的囚禁为起点,介绍了离子在离子阱中的经典运动、量子运动及离子的光场耦合,随后以镱离子量子计算平台为核心,讲述了离子量子比特的电离、冷却、初始化、探测、纠缠建立、比特控制、通用量子门构建等内容。通过对离子阱量子计算系统的学习,更加深刻地认识到量子计算系统对测控系统实时、拓展性等方面的切实需求。
在第\ref{section:fpga_rtmq}章中介绍了一种专为量子物理实验设计的强实时、可拓展、分布式的RTMQ量子测控系统和其基于FPGA的实现,同时介绍了其配套设计的指令集以及通信链路。随后介绍了RTMQ系统的一些重要功能外设的FPGA实现,其中重点介绍了高速通用数字PID、高速通用数字IIR滤波器等数字化系统模块的实现,为随后各类控制系统的实现奠定了基础。RTMQ测控系统相对于现有的量子测控系统所具有的显著优势是其严格的实时性、分布式计算处理能力以及强大的可拓展性,具有广阔的应用前景。
在第\ref{section:helical}章中介绍了离子阱系统中离子囚禁相关的一种关键的微波器件——螺线管谐振腔,通过基于有限元方法的仿真软件HFSS对谐振腔进行了建模和仿真,预实验结果取得了很好的一致性,同时建立了更为准确的谐振腔数学模型来描述谐振腔的特性。除此之外,也对谐振腔进行了优化设计,在谐振腔的稳定性、耦合方便性、模块化等方面有了很大的提升。
在前述内容的基础上,接着在第\ref{section:implementation}章中给出了基于RTMQ量子测控系统实现的若干离子阱量子计算子系统,包括通过稳定谐振腔的输出来实现的离子阱频率稳定子系统、针对激光进行的激光功率稳定和脉冲激光拍频稳定等的子系统搭建和测试。通过使用RTMQ系统将更多核心模块数字化的方式及大地提高了系统的灵活性和稳定性。

% 本文的重点在于量子比特的控制,以测控系统为中心主要关注离子比特控制系统中电子学和光学相关的子系统搭建问题。在上述测控系统、囚禁系统、光学稳定系统、光学操控系统的基础上可以进一步结合其它如离子电离、离子囚禁、离子冷却等整合实现离子量子比特门。虽然由于实验室整体方向规划和进度的原因,未能最终实现量子门操作,但在与门操作保真度直接相关的实验参数提升和稳定方面做了相当多的系统性工作,包括螺旋谐振腔设计方法的研究和改进以及阱频率、激光功率等参数的稳定方案,这些成果对于囚禁离子量子计算领域都具有普遍的实用意义。实际搭建中的脉冲光操控方案的离子阱系统如\ref{fig:ion_trap_system}图所示。进一步再结合光路寻址系统等可以实现多离子比特门的控制,最终面向通用量子计算计算机的实现。值得一提的是,芯片阱的出现使得离子阱的规模化、集成化和小型化的进程大大加快,是离子阱领域重要的研究方向之一,也是发展趋势所在,我们也正在开展这方面的探索和尝试。


\end{conclusion}


