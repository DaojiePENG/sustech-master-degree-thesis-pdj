% !TeX root = ../sustechthesis-example.tex

\begin{conclusion}

量子计算拥有着广阔的发展和应用前景。离子阱量子计算因其较高的相干时间和保真度是量子计算未来重要的发展方向之一。离子阱系统是一个综合复杂的系统,其中测控系统占有十分关键的位置。本文主要研究内容为面向离子阱量子计算的实验测控系统,介绍了与之相关的离子阱量子计算物理背景、专为量子实验设计的RTMQ测控系统、用于离子阱的螺线管谐振腔以及基于前述内容实现的若干重要量子计算子系统。

第\ref{section:quantum_computation}章介绍了量子测控系统的重要研究背景基础。以离子的囚禁为起点,介绍了离子在离子阱中的经典运动、量子运动及离子的光场耦合,随后以镱离子量子计算平台为核心,讲述了离子量子比特的电离、冷却、初始化、探测、纠缠建立、比特控制、通用量子门构建等内容。通过对离子阱量子计算系统的学习,更加深刻地认识到量子计算系统对测控系统实时、拓展性等方面的切实需求。
在第\ref{section:fpga_rtmq}章中介绍了一种专为量子物理实验设计的强实时、可拓展、分布式的RTMQ量子测控系统和其基于FPGA的实现,同时介绍了其配套设计的指令集以及通信链路。RTMQ测控系统相对于现有的量子测控系统所具有的显著优势是其严格的实时性、分布式计算处理能力以及强大的可拓展性,具有广阔的应用前景。
在第\ref{section:helical}章中介绍了离子阱系统中离子囚禁相关的一种关键的微波器件——螺线管谐振腔,通过基于有限元方法的仿真软件HFSS对谐振腔进行了建模和仿真,预实验结果取得了很好的一致性,同时建立了更为准确的谐振腔数学模型来描述谐振腔的特性。除此之外,也对谐振腔进行了优化设计,在谐振腔的稳定性、耦合方便性、模块化等方面有了很大的提升。
在前述内容的基础上,接着在第\ref{section:implementation}章中给出了RTMQ系统的测控板硬件设计及一些芯片选型,同时也给出了一些基本的基于RTMQ系统的外设拓展的设计和PFGA实现。基于RTMQ量子测控系统实现的若干离子阱量子计算子系统,包括通过稳定谐振腔的输出来实现的离子阱频率稳定子系统、针对激光进行的激光功率稳定和脉冲激光拍频稳定等的子系统搭建和测试。特别地,在系统搭建过程中也引入了高速通用数字PID和高速数字通用IIR滤波器作为RTMQ系统的重要外设拓展。

面向离子量子计算的实验需求,本研究通过使用RTMQ量子测控系统将更多核心模块数字化的方式及大地提高了系统的集成度、灵活性和稳定性,为未来大规模通用量子计算奠定了测控方面的重要基础。

\end{conclusion}


