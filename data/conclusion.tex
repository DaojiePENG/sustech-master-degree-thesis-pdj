% !TeX root = ../sustechthesis-example.tex

\begin{conclusion}

离子阱量子计算因其较高的相干时间和保真度是量子计算未来重要的发展方向之一。离子阱系统是一个复杂的综合系统,其中测控系统占有十分关键的位置。随着量子计算技术的不断发展,传统测控系统以及早些年开发的量子测控系统逐渐难以适应日益增高的实时性、拓展性、集成度等量子测控需求。此外,在离子阱系统中,由于缺乏对谐振腔几何结构对频率Q值等的具体影响和耦合过程的认识,以及当前谐振腔经验公式的不准确性和装配结构的不稳定性,高效准确地设计和使用谐振腔仍存在挑战。
为应对这些挑战,本文开展了面向离子阱量子计算的实验测控系统研究,主要的研究成果和贡献如下:
\begin{enumerate}
    \item 引入和分析了RTMQ测控系统架构,以满足量子物理实验对测控系统的实时性、拓展性、集成度等方面的需求。基于囚禁离子量子比特制备和控制的理论背景分析了量子物理实验对测控系统的实时性、可拓展性、集成性等方面的切实需求。针对这些需求介绍了一种面向离子量子计算系统的强实时、分布式、易拓展的RTMQ测控系统架构,分析了RTMQ的系统架构、内部模块、配套的汇编指令集及多节点间实时通信链路系统。
    相对于传统测控和现有的量子测控系统,RTMQ测控系统架构所具有的显著优势是其严格的实时性、分布式计算处理能力以及强大的可拓展性;
    \item 对离子阱系统中的螺线管谐振腔进行了系统的仿真、实验、建模、结构优化研究,以更准确高效地设计高Q谐振腔。使用HFSS建立了谐振腔的3D仿真模型,采用有限元的方式进行了射频驱动模式下的仿真,并通过实验制作测量与仿真结果数据的对比验证了仿真-实验结果一致性。基于仿真实验结果分析了主线圈、输出线、屏蔽外壳、耦合线圈等4项几何参数对谐振腔性能的影响并阐述了其使用耦合线圈进行耦合过程的特性。通过经验公式和物理分析的方式建立了谐振腔的数学模型并总结了设计特定频率下的高Q谐振腔的方法。针对用于离子阱系统的谐振腔结合结构进行了优化设计,使之稳定性更高、更易装配和耦合。
    \item 设计并制作了1套适用于32位RTMQ系统的测控板硬件及4个芯片管理固件、2个功能外设拓展以满足离子阱量子计算对实时性、拓展性、集成度等方面的需求。基于FPGA设计并制作了1套适用于32位RTMQ系统的测控板硬件,该板卡工作频率为200 MHz,时间分辨率5ns。单板集成了2路32位DDS、2路16位ADC、2路16位DAC、并提供64路GPIO和32对CPCI LVDS用户接口。用Verilog语言设计了UART、SPI、AD9910、ADC等4个芯片管理固件以及RTMQ系统的2个重要功能外设拓展固件——板上工作频率可达200 MHz以上的32位高速通用数字PID和板上工作频率可达25MHz以上的32位高速通用数字IIR滤波器。相对于传统计算机+模拟器件的测控系统,本系统拥有着集成度高、稳定性好、可拓展性强、操作灵活等优势。
    \item 结合螺线管谐振腔等器件搭建了3个离子阱关键子的系统对这套RTMQ测控板卡进行了测试,验证了RTMQ量子测控系统的有效性和优越性。面向离子阱量子计算对射频信号、激光信号等稳定性的实际应用需求,本研究使用这套RTMQ测控系统搭建和测试了离子阱频率稳定、激光幅度稳定和激光拍频稳定等3个离子阱关键子系统,预期可将相关噪声对单比特$\theta=\pi$翻转操作保真度的影响分别由稳定前的$7.994\times10^{-6}$、$1.385\times 10^{-4}$、$4.992\times 10^{-1}$,降低到了稳定后的$3.734\times10^{-7}$、$1.646\times 10^{-6}$、$1.440\times 10^{-4}$。取得了十分显著的效果,能够很好地满足量子物理实验系统使用需求。
  \end{enumerate}

% \begin{enumerate}
%     \item 引入了一种强实时性、可拓展性好、高度数字化和集成化的RTMQ测控系统架构来满足量子物理实验对测控系统更高的需求并对其进行了拓展。分析了RTMQ的系统架构、内部模块、配套的指令集及多节点的实时通信链路系统。相对于传统测控和现有的量子测控系统,RTMQ测控系统架构所具有的显著优势是其严格的实时性、分布式计算处理能力以及强大的可拓展性;
%     \item 研究了离子囚禁系统中的重要射频器件——螺线管谐振腔。以更高的Q值和更稳定易用的设计为目标对螺线管谐振腔进行了仿真、实验、建模、设计等方面的优化研究。通过基于有限元方法的仿真软件HFSS对谐振腔进行了建模和仿真,与实验结果取得了很好的一致性,同时建立了更为准确的谐振腔数学模型来描述谐振腔的特性。对谐振腔的装配结构进行了优化设计,在谐振腔的稳定性、耦合方便性、模块化等方面有了很大的提升;
%     \item 设计了一套基于RTMQ架构的测控板硬件,板上集成了若干数字芯片可用于对离子量子计算系统常用的模拟系统进行数字化升级,并基于FPGA实现了与之匹配的重要功能外设固件。本套RTMQ系统凭借其数字化的优势极大地简化整个实验平台仪器的使用,使实验过程有着更强的可控性和更好的自动化水平;
%     \item 基于RTMQ测控板卡搭建并测试了离子量子计算系统中的几个关键子系统,以提高离子量子比特操作保真度:离子阱频率稳定系统、激光幅度稳定系统和激光拍频稳定系统,特别地,在系统搭建过程中也引入了高速通用数字PID和高速数字通用IIR滤波器作为RTMQ系统的重要功能外设拓展。这些系统为实现集成化、数字化、低成本、灵活稳定的离子量子计算系统奠定了基础。
% \end{enumerate}


% 第\ref{section:quantum_computation}章介绍了量子测控系统的重要研究背景基础。以离子的囚禁为起点,介绍了离子在离子阱中的经典运动、量子运动及离子的光场耦合,随后以镱离子量子计算平台为核心,讲述了离子量子比特的电离、冷却、初始化、探测、纠缠建立、比特控制、通用量子门构建等内容。通过对离子阱量子计算系统的学习,更加深刻地认识到量子计算系统对测控系统实时、拓展性等方面的切实需求。
% 在第\ref{section:fpga_rtmq}章中介绍了一种专为量子物理实验设计的强实时、可拓展、分布式的RTMQ量子测控系统,同时介绍了其配套设计的指令集以及通信链路。RTMQ测控系统相对于现有的量子测控系统所具有的显著优势是其严格的实时性、分布式计算处理能力以及强大的可拓展性,具有广阔的应用前景。
% 在第\ref{section:helical}章中介绍了离子阱系统中离子囚禁相关的一种关键的微波器件——螺线管谐振腔,通过基于有限元方法的仿真软件HFSS对谐振腔进行了建模和仿真,预实验结果取得了很好的一致性,同时建立了更为准确的谐振腔数学模型来描述谐振腔的特性。除此之外,也对谐振腔进行了优化设计,在谐振腔的稳定性、耦合方便性、模块化等方面有了很大的提升。
% 在前述内容的基础上,接着在第\ref{section:implementation}章中给出了RTMQ系统的测控板硬件设计及一些芯片选型,同时也给出了一些基本的基于RTMQ系统的外设拓展的设计和PFGA固件实现。
% 并且基于这套RTMQ量子测控系统实现了若干离子阱量子计算子系统的搭建和测试,包括通过稳定谐振腔的输出来实现的离子阱频率稳定子系统、针对激光进行的激光幅度稳定和脉冲激光拍频稳定系统等。特别地,在系统搭建过程中也引入了高速通用数字PID和高速数字通用IIR滤波器作为RTMQ系统的重要功能外设拓展。RTMQ系统凭借其数字化的优势而可以极大地简化整个实验平台仪器的使用,使实验过程有着更强的可控性和更好的自动化水平。

面向离子量子计算的实验需求,本研究通过仿真、实验、建模、结构优化等方法使设计和使用高Q螺线管谐振腔更加高效和稳定,实现的RTMQ量子测控板卡系统通过将更多核心模块数字化的方式,极大地提高了系统的实时性、集成度、灵活性和稳定性,推动了整个离子阱量子计算系统的数字化、集成化进程,为未来大规模通用量子计算奠定了测控方面的重要基础。
当前构建的RTMQ实验测控系统是基于FPGA的,受限于FPGA芯片,其工作频率为200 MHz,时间分辨率为5ns,接下来一个重要的探索方向是通过进一步的IC设计来使得该系统能工作在更高的频率以获得更高的时间分辨率。而且,本研究所涉及的仅为单板卡的研究,实际大规模离子阱量子计算系统中需要成百上千个板卡协同工作,因此多板卡之间的时序对齐、任务分发、数据处理等有待更进一步研究。此外,量子测控系统未来的发展方向还包括:更多功能的板上集成、测控设备小型化、专门化、用户友好以及成本的降低等。


\end{conclusion}


