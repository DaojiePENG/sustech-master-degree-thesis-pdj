% !TeX root = ../sustechthesis-example.tex

\begin{acknowledgements}
  衷心感谢导师张君华副研究员和量子院路尧、王钊副研究员、张丽昀助理教授等对本人的精心指导,老师们的言传身教将使我终生受益。

  依稀记得刚来实验室时君华老师对我的真切关怀,那时候对深圳这座城市完全是陌生的,对将要从事的离子阱量子计算领域的研究也是半知半解的。十分感谢君华老师带我熟悉深圳和实验室的环境,并且给我推荐了如《量子力学概论》、《量子信息》、《Atom》等领域相关的书籍作为入门学习。这些书籍为我随后在离子阱量子计算方向的研究打下了很好的基础。与此同时,路尧老师在经常会在实验室白板上分享有关光与离子相互作用的知识。这些知识十分具体,将从前在书本上学习到的许多量子的概念结合离子阱这个系统进行了具象化。路老师的这些分享帮助了我快速熟悉和了解离子阱量子计算整个系统的原理和实验,使我受益匪浅。在研一下学期到研二期间,十分荣幸能够与王钊老师一起合作关于用于离子阱系统的螺线圈谐振腔的研究。这是我着手的第一个真正意上的研究工作。我们从文献阅读到仿真模拟,再到实验制作与验证以及随后的理论模型的建立等等。在此期间王老师给了我许多有益的建议和帮助,比如在实验前要做好充分的准备、每次实验要有良好的计划和预期、如何合理规划实验与数据处理的时间等等。与王老师的合作初步建立起了我对科研项目的整体认识。在研二和三期间与君华老师、路尧老师、王钊老师一起进行脉冲激光实现离子量子门方案的实验平台的过程中,君华老师在RTMQ系统的开发和使用上给了我很大的支持,包括检查代码问题、测控板问题等等;路尧老师和王钊老师分别在整体系统架构和原理方面,以及离子的囚禁和真空系统方面给了我很大的帮助。尽管受到各种限制没能最终完全实现基于脉冲激光的离子量子门的整个系统的搭建和测试而有些遗憾,在搭建诸如激光功率稳定、脉冲激光拍频锁定、捕捉离子、测控系统开发、谐振腔等等重要的子系统的过程中也学到了许多宝贵的知识,积累了丰富的经验,为以后的研究打下了坚实的基础。在整个研究期间,张丽昀老师在实验室布局、器件采购和实验设备的使用上一直细心耐心地安排,给了我许多指导。
  
  作为一个新的实验组,组里的各位老师都朝气磅礴,作为组里的第一名学生,我深受各位老师们的感染。老师们在离子阱量子计算领域专业的素养、乐于分享的精神、乐观开朗的态度、精益求精的品质等等,在科研过程中一直都潜移默化地影响着我,使我取得了巨大的成长和进步。再次感谢各位老师的引导和陪伴!

  % 在美国麻省理工学院化学系进行九个月的合作研究期间,承蒙 Robert Field 教授热心指导与帮助,不胜感激。

  % 感谢×××××实验室主任×××教授,以及实验室全体老师和同窗们学的热情帮助和支持!

  最后,感谢父母、家人还有各位同学朋友的关爱和支持,亲人的期望和祝福是我前进的不竭动力,同学朋友们的陪伴和交流使我的研究生生活更加丰富多彩。感谢一路有你们,你们永远在我心中!

  % 本课题承蒙国家自然科学基金资助,特此致谢。
\end{acknowledgements}
