% !TeX root = ../sustechthesis-example.tex

\begin{acknowledgements}
  衷心感谢研究生期间导师张君华副研究员和路尧副研究员、王钊副研究员、张丽昀助理教授、汤云翠秘书、高振副教授、刘明教授等对本人的精心指导和帮助,老师们的言传身教使我终生受益。也非常感谢母校南方科技大学和国际量子研究院为我提供十分优越的平台环境开展学习和研究,让我度过了一段充实快乐的时光。

  依稀记得刚来量子院离子阱量子计算实验室时君华老师对我的真切关怀,那时候对深圳这座城市完全是陌生的,对将要从事的离子阱量子计算领域的研究也是半知半解的。十分感谢君华老师带我熟悉深圳和实验室的环境,并且给我推荐了如《量子力学概论》、《量子信息》、《Atom》等领域相关的书籍作为入门学习。这些书籍为我随后在离子阱量子计算方向的研究打下了很好的基础。与此同时,路尧老师经常会在实验室白板上分享有关光与离子相互作用的知识。这些知识十分具体,将从前在书本上学习到的许多量子的概念结合离子阱这个系统进行了具象化。路老师的这些分享帮助了我快速熟悉和了解离子阱量子计算整个系统的原理和实验,使我受益匪浅。
  
  在研一下学期到研二期间,十分荣幸能够与王钊老师一起合作关于用于离子阱系统的螺线圈谐振腔的研究。这是我着手的第一个真正意上的研究工作。我们从文献阅读到仿真模拟,再到实验制作与验证以及随后的理论模型的建立等等。在此期间王老师给了我许多有益的建议和帮助,比如在实验前要做好充分的准备、每次实验要有良好的计划和预期、如何合理规划实验与数据处理的时间等等。与王老师的合作初步建立起了我对科研项目的整体认识。也很有幸在高等电磁学课上与高振老师结识,高老师用哲学和科学相结合的给了我许多科研上和学习生活上的启发,与高老师的每次交流总能受益匪浅。
  
  在研二和三期间与君华老师、路尧老师、王钊老师一起进行脉冲激光实现离子量子门方案的实验平台的过程中,君华老师在RTMQ量子测控系统的开发和使用上给了我很大的支持,包括芯片选型意见、Verilog代码问题、测控板问题等等;路尧老师和王钊老师分别在整体系统架构和原理方面,以及离子的囚禁和真空系统方面给了我很大的帮助。
  在进行诸如激光幅度稳定、脉冲激光拍频锁定、测控系统设计开发、捕捉离子、谐振腔等离子量子计算重要的子系统的研究过程中我学到了许多宝贵的知识,积累了丰富的经验,为以后的研究打下了坚实的基础。
  在整个研究期间,张丽昀老师在实验室布局、器件采购和实验设备的使用上一直细心耐心地安排,给了我许多指导;课题组秘书汤云翠老师为我与学校和学院的沟通提供了很大的帮助,一直及时通知并尽心尽力地协助我进行各项事务的办理。
  
  此外,研三期间在香港科技大学(广州)ROAS学域进行的交流学习期间,承蒙刘明教授热心指导和支持,不胜感激。
  
  作为一个新的实验组,组里的各位老师都朝气磅礴,作为组里的第一名学生,我深受各位老师们的感染。老师们在离子阱量子计算领域专业的素养、乐于分享的精神、乐观开朗的态度、精益求精的品质等等,在科研过程中一直都潜移默化地影响着我,使我取得了巨大的成长和进步。再次感谢各位老师的引导和陪伴!

  最后,感谢父母亲人、女友还有各位同学朋友,父母亲人的期望和关爱是我前进的不竭动力,女友的支持和陪伴使我的信念更加坚定,同学朋友们的相处和交流让我的研究生生活更加丰富多彩。感谢一路有你们,你们永远在我心中!

  % 本课题承蒙国家自然科学基金资助,特此致谢。
\end{acknowledgements}
