% !TeX root = ../sustechthesis-example.tex

\chapter[囚禁离子量子比特]{囚禁离子量子比特\label{section:quantum_computation}}
% \textcolor{red}{
% 介绍离子在离子阱中的经典和量子运动、介绍离子的一些特别的运动量子态;
% 讲一下离子量子比特的主要特点(一致性、全连接性、相干时间长等)、介绍一下单比特门操作的主要方案(微波、激光拉曼);离子的运动不用讲太多,先讲囚禁,再讲一下运动的量子化、拉曼光对运动的耦合以及M-S门就行,特殊量子态可以不用讲,或至少不用大篇幅。这里M-S门部分可以给后面Helical部分做点铺垫,讲一下阱频率对门操作保真度和操作时间的影响。
% 整体而言,离子量子比特的量子态制备、探测和单比特门一个小节,运动部分再2个小节,差不多。
% }
量子测控是量子计算系统中的重要组成部分,相对于传统测控系统来说,量子测控系统的特别之处在于它不仅需要具备经典场景下的信息处理能力,还需要具备量子场景下的信息处理能力。
% 总体来说,量子物理场景对测控系统提出了实时性、可拓展性、信息处理方式等方面的新需求。
对于离子量子计算来说,具体需求的认识和处理都需要我们对目标系统拥有充分的了解,这样才能更好地开发、优化和部署适宜的测控系统。
本章内容将介绍与量子测控系统紧密相关的离子量子计算的重要背景以分析和理解其切实需求,内容包括离子的囚禁及其涉及到的经典和量子运动,特别地,针对当前实验室所使用的镱离子介绍它的能级结构、比特编码、比特控制,以及基于离子构建的通用量子门。


\section[离子的囚禁及在阱中的经典和量子运动]{离子的囚禁及在阱中的经典和量子运动\label{section:ion_trap_motion}}
离子阱量子计算的一个最基本的问题就是获得在空间中稳定存在的离子,利用其具备的量子特性来作为量子计算的比特载体。离子阱采用射频阱对离子进行动态囚禁,最具代表性的一类是四极阱,它及它的一些变型也是目前在离子阱量子计算研究中应用最广泛的一类离子阱。四极阱的电势描述如下:
\begin{align}
    % \label{eq:quadrupolar_trap_potential}
    \Phi(x, y, z, t) = &U\frac{1}{2}(\alpha x^2 + \beta y^2 + \gamma z^2) \label{eq:time_independent_part}\\
    \label{eq:time_dependent_part}
    &+ \tilde{U}\cos (\omega_{rf}t)\frac{1}{2}(\alpha ' x^2 + \beta ' y^2 + \gamma ' z^2) 
\end{align}

其中$\Phi(x, y, z, t)$是四极阱的电势,$U$和$\tilde{U}$分别是静态和动态信号的幅值,$\omega_{rf}$是动态信号的频率,$\alpha, \beta, \gamma$及$\alpha', \beta', \gamma'$分别是静态和动态信号各方向上的系数。公式\eqref{eq:time_independent_part}是不依赖时间的项,公式\eqref{eq:time_dependent_part}是随时间变化的项。整个电势的表达式每时每刻都要满足拉普拉斯方程(Laplace equation)$\Delta \Phi=0$的约束条件,从中可以导出整个四极阱的几何参数约束:
\begin{align}
    \alpha + \beta + \gamma =0,\\
    \alpha ' + \beta ' + \gamma ' =0
\end{align}

其中各参数在公式\eqref{eq:time_independent_part}和公式\eqref{eq:time_dependent_part}中定义。从这些限制可以明显看出,在自由空间中电势不可能稳定地产生局部三维最小值,因此电势只可能以动态方式来对离子进行囚禁。通过选择合适的四极阱几何参数,再结合适当的驱动射频的频率和驱动电压我们可以做到这一点,其中一种几何参数选择如下:
\begin{align}
    -(\alpha + \beta )= \gamma > 0,\\
    \alpha ' = - \beta '
\end{align}

这种几何参数的设置会使离子在$x,y$平面上动态地被囚禁,在$z$方向上静态地被囚禁。在这种设置的离子阱中,多个离子会沿着$z$轴形成线性的离子链,这便是人们所知的\emph{线性离子阱(Linear Trap)},也被称为\emph{Paul Trap}\cite[]{Paul_1990}。


\subsection[离子在RF阱中的经典运动]{离子在RF阱中的经典运动\label{section:ion_classical_motion}}

在接下来的两小节里将介绍囚禁离子的经典运动方程及其解析解(第\ref{section:classical_motion}节),并给出这些解的低阶近似(第\ref{section:lowest_order_approximation}节)。

\subsubsection[经典运动方程]{经典运动方程\label{section:classical_motion}}

一个质量为$m$电荷量为$Z|e|$的粒子在如公式\eqref{eq:time_independent_part}所描述的电场中的经典运动方程由Paul等人\cite[p415]{Paul1958}给出。粒子的运动在空间坐标方向上中是解耦的。下面只讨论$x$方向上的运动;其他方向可以类似地处理。运动方程如下:
\begin{align}
    \ddot{x}=-\frac{Z|e|}{m}\frac{\partial \Phi}{\partial x}=-\frac{Z|e|}{m}[U\alpha + \tilde{U}\cos(\omega_{rf}t)\alpha ']x
\end{align}

经过下面的参数代换,这个方程可以转化为标准的\emph{马修方程(Mathieu Equation, ME)}形式:
\begin{align}
    \frac{d^2x}{d\xi^2}+[a_x-2q_x\cos(2\xi)]x=0\label{eq:mathieu_equation}
\end{align}

相应的参数代换为:
\begin{align}
    \xi=\frac{\omega_{rf}t}{2},\ a_x=\frac{4Z|e|U\alpha}{m\omega_{rf}^2},\ q_x=\frac{2Z|e|\tilde{U}\alpha '}{m\omega_{rf}^2}\label{eq:parameters_substitution}
\end{align}

ME方程属于一般的周期系数微分方程。它的稳定解的一般形式可以由\emph{弗洛奎定理(Floquet Theorem)}导出\cite[]{McLachlan, McQuarrie}:
\begin{align}
    x(\xi)=&Ae^{i\beta_x\xi}\sum_{n=-\infty}^{\infty}C_{2n}e^{i2n\xi}\\
    &+ Be^{-i\beta_x\xi}\sum_{n=-\infty}^{\infty}C_{2n}e^{-i2n\xi}\label{eq:mathieu_solution}
\end{align}

其中实值特征指数$\beta_x$和系数$C_{2n}$仅是$a_x$和$q_x$的函数,不依赖于初始条件。$A$和$B$是任意常数,可用于满足边界条件或规格化特解。将公式\eqref{eq:mathieu_solution}代入公式\eqref{eq:mathieu_equation}可以得到一个递归关系:
\begin{align}
    C_{2n+2}-D_{2n}C_{2n}+C_{2n-2}=0,\\
    D_{2n}=[a_x-(2n+\beta_x)^2]/q_x\label{eq:recursion_raltion}
\end{align}

这一递归关系将实值特征指数$\beta_x$、系数$C_{2n}$与$a_x$、$q_x$联系起来。通过进一步地整理也可以得到$C_{2n}$的表达式:
\begin{align}
    C_{2n+2}=\frac{C_{2n}}{D_{2n}-\frac{1}{D_{2n+2}-\frac{1}{\dots}}}\\
    C_{2n}=\frac{C_{2n-2}}{D_{2n}-\frac{1}{D_{2n-2}-\frac{1}{\dots}}}\label{eq:c_2n_fraction}
\end{align}

利用结合上述公式$\beta_x$也可以计算:
\begin{align}
    \beta_x^2=a_x-q_x\left(\frac{1}{D_0-\frac{1}{D_2-\frac{1}{\dots}}} + \frac{1}{D_0-\frac{1}{D_{-2}-\frac{1}{\dots}}}\right) \label{eq:beta_x_fraction}
\end{align}

可以根据所需的精度,选择截断公式\eqref{eq:c_2n_fraction}和公式\eqref{eq:beta_x_fraction}中的连分式来获取相应的结果。
实际上,对于实验中常用到的典型$a_x$和$q_x$值,连分式中高阶项的贡献会迅速下降。

% \textcolor{red}{这里的$a_x, q_x$具体的含义后面有时间了可以再看看。}

\subsubsection[低阶近似]{低阶近似\label{section:lowest_order_approximation}}
实际实验系统中采用的在公式\eqref{eq:parameters_substitution}中定义的参数往往是满足$(|a_x|,q_x^2)\ll 1$的。在此条件下,假设$C_{\pm 4}\simeq 0$,则可以得到$x(t)$轨迹的\emph{低阶近似(Lowest-order Approximation)}。再同时设置初始条件$A=B$,公式\ref{eq:recursion_raltion}可以得到:
\begin{align}
    \beta_x\approx \sqrt{a_x+q_x^2/2},\\
    x(t)\approx2AC_0\cos\left(\beta_x\frac{\omega_{rf}}{2}t\right)\left[1-\frac{q_x}{2}\cos(\omega_{rf}t)\right]\label{eq:classical_motion_solution}
\end{align}

囚禁离子在$x$方向上的轨迹$x(t)$的由频率为$\nu=\beta_x\omega_{rf}/2\ll \omega_{rf}$的简谐振荡叠加频率为$\omega_{rf}$的RF频率造成的\emph{受迫运动}组成,分别称为\emph{长期运动(Secular-motion)}和\emph{微运动(Micro-motion)}两者相位相差$180^\circ$;
离子在离子阱中的微运动的频率为$\omega_{rf}\gg \nu$,且其振幅为长期运动振幅的$q_x/2\ll 1$,这也是它被称为微运动的原因。如果忽略微运动,则离子的运动可以近似为频率为$\nu$的谐振子的运动。
% 在大多数情况下,如果离子处于相当低的动能,即使我们用量子力学的方法来处理离子的质心运动,这个处理也是合理的。


\subsection[离子在RF阱中的量子力学运动]{离子在RF阱中的量子力学运动\label{section:quantum_motion}}
% \textcolor{red}{主要参考文献\cite[chap2-B]{Leibfried_Blatt_Monroe_Wineland_2003}}

如第\ref{section:ion_classical_motion}节中所述,经典的运动分析似乎已经可以很好地描述离子在离子阱中的运动了。但是,离子的冷却过程以及离子的边带跃迁等都涉及量子过程,因此需要了解离子运动的量子力学图景。
% 由于四极阱产生的囚禁势场不是静态的而是与时间相关的,因此不能理所当然地认为在有效时间平均势中量化运动已经为我们提供了囚禁离子足够的图景。实际上,在离子阱的实验中,即使是对离子阱中的冷却过程的简单解释,以及对非经典状态的描述,也都依赖于运动的量子力学图景的。
在接下来的两小节里,我将根据文献\cite[]{Arimondo_Phillips_Strumia_1992}中的方法导出囚禁离子在射频场中的量子力学表述,并讨论在实验中使用的囚禁参数范围内囚禁离子的量子化运动用静态谐振子的近似。

\subsubsection[量子力学运动方程]{量子力学运动方程}
对于囚禁离子运动的量子力学处理,我们假设与时间相关的势在囚禁离子质心的三个笛卡尔坐标中的每一个中都是二次的(一维谐振子势\cite[]{Solimeno_Di_Porto_Crosignani})。然后,与经典运动一样,问题可分为三个一维问题。在一维中,用各自的算子$\hat{x}$替换坐标$x$,于是可以将与时间相关的势$V(T)$写为:
\begin{align}
    V(t)=\frac{m}{2}W(t)\hat{x}^2
\end{align}

其中,
\begin{align}
    W(t)=\frac{\omega_{rf}^2}{4}\left[a_x+2q_x\cos(\omega_{rf}t)\right]
\end{align}

可以被认为是一个时变弹簧常数,它的作用类似于在静态势谐振子中$\omega^2$的作用。在以上的定义下,囚禁离子运动的哈密顿量$H^{(m)}$的形式和我们在量子力学中处理的静态谐振子的哈密顿量很相似:
\begin{align}
    \hat{H}^{(m)}=\frac{\hat{p}^2}{2m}+\frac{m}{2}W(t)\hat{x}^2\label{eq:static_harmiltonian_oscillator}
\end{align}

于是我们可以很轻松地写出这些运动算子在\emph{海森堡图景(Heisenberg Picture)}下的的方程:
\begin{align}
    \dot{\hat{x}}= \frac{1}{i\hbar}\left[\hat{x,\hat{H}^{(m)}}=\frac{\hat{p}}{m}\right],\\
    \dot{\hat{p}}= \frac{1}{i\hbar}\left[\hat{p},\hat{H}^{(m)}\right]=-mW(t)\hat{x}
\end{align}

他们的一个更紧凑的方程形式如下:
\begin{align}
    \ddot{\hat{x}}+W(t)\hat{x}=0 \label{eq:quantum_motion_equation}
\end{align}

如果用函数$u(t)$替换算子$\hat{x}$,我们可以很容易验证这个公式\eqref{eq:quantum_motion_equation}与\emph{马修方程}\eqref{eq:mathieu_equation}是等价的。这就是我们能够借助前面所叙述的马修方程的解来寻找公式\eqref{eq:quantum_motion_equation}的解。添加边界条件:
\begin{align}
    u(0)=1,\ \hat{u}(0)=i\nu \label{eq:boundary_condition}
\end{align}

这对应于公式\eqref{eq:mathieu_solution}中的$A=1,\ B=0$,可以得到:
\begin{align}
    u(t)=e^{i\beta_x\omega_{rf}/2}\sum_{n=-\infty}^{\infty} C_{2n}e^{i n\omega_{rf}t}\equiv e^{i\beta_x\omega_{rf}t/2}\Phi(t) \label{eq:ut_expression}
\end{align}

其中$\Phi(t)$是一个周期为$T=2\pi/\omega_{rf}$的周期函数。于是公式\eqref{eq:boundary_condition}变为:
\begin{align}
    u(0)=\sum_{n=-\infty}^{\infty}C_{2n}=1,\ \nu = \omega_{rf}\sum_{n=-\infty}^{\infty}C_{2n}(\beta_x/2+n)
\end{align}

这个解及其复共轭是线性独立的;因此,它们服从\emph{Wronskian恒等式}:
\begin{align}
    u^*(t)\dot{u}(t)-u(t)\dot{u}^*(t)=u^*(0)\dot{u}(0)-u(0)\dot{u}^*(0)=2 i \mu
\end{align}

未知坐标$\hat{x}(t)$和$u(t)$满足相同的微分方程,因此复杂的线性组合:
\begin{align}
    \hat{C}(t)=\sqrt{\frac{m}{2\hbar \nu}}i\left\{u(t)\dot{\hat{x}(t)-\dot{u}(t)\hat{x}(t)}\right\} \label{eq:complex_combination}
\end{align}

与其如下的Wronskian恒等式成正比,并且在时间上也是恒定的:
\begin{align}
    \hat{C}(t)=\hat{C}(0)=\frac{1}{\sqrt{2m \hbar \nu}}\left[m\nu\hat{x}(0)+i\hat{p}(0)\right]
\end{align}

此外,等式右边恰好是质量$m$和频率$\nu$的在静态谐振子势场中的湮灭算符:
\begin{align}
    \hat{C}(t)=\hat{C}(0)=\hat{a}
\end{align}

也就是说有如下式:
\begin{align}
    \left[\hat{C},\hat{C}^\dagger\right]=\left[\hat{a},\hat{a}^\dagger\right]=1
\end{align}

这个静态势场中的谐振子在后续将被称为\emph{参考谐振子(Reference Oscillator)}。

海森堡算符$\hat{x}(t)$和$\hat{x}(t)$可以用$u(T)$和参考谐振子的算符用公式\eqref{eq:complex_combination}重新表示:
\begin{align}
    \hat{x}(t)=\sqrt{\frac{\hbar}{2m\nu}}\left\{\hat{a}u^*(t)+\hat{a}^\dagger u(t)\right\},\\
    \hat{p}(t)=\sqrt{\frac{\hbar m}{2\nu}}\left\{\hat{a}\dot{u}^*(t)+\hat{a}^\dagger \dot{u}(t)\right\}
\end{align}

所以囚禁离子的整个时间依赖性由特殊解$u(t)$及其复共轭给出。
这样一来,对于接下来的计算,在海森堡图景中表述一些列的时间依赖波函数就很方便了。同样,上面使用的参考谐振子也将非常有帮助。与静态势的情况类似,我们将考虑一系列的基态$\ket{n,t}$,其中$n=1,2,\dots,\infty$,这些状态被称为谐振子\emph{数态(Fock States)}。参考谐振子$\ket{n=0}_\nu$的基态满足条件:
\begin{align}
    \hat{a}\ket{n=0}_\nu=\hat{C}(t)\ket{n=0}_\nu=0 \label{eq:obey_condition}
\end{align}

由于海森堡算子$\hat{C}$是通过$\hat{C}(t)=\hat{U}^\dagger(t)\hat{C}_S\hat{U}(t)$与$\hat{C}_S$联系起来的,我们可以很快得到(其中$\hat{U}(t)=\exp{\left[-(i/\hbar)\hat{H}^{(m)}\right]}$):
\begin{align}
    \hat{C}_S(t)\hat{U}(t)\ket{n=0}_\nu=\hat{C}_S(t)\ket{n=0,t}=0 \label{eq:oscillator_condition}
\end{align}

只需要通过将公式\eqref{eq:obey_condition}左侧与$\hat{U}(t)$相乘,并注意到$\hat{U}(t)\ket{n=0}_\nu$是从静态潜在参考谐振子的基态演变而来的时间相关谐振子的薛定谔态。由于薛定谔算子$C_S(t)$的时间依赖完全取决于$u(t)$的时间演化,于是公式\eqref{eq:oscillator_condition}等价于:
\begin{align}
    \left[u(t)\hat{p}-m\dot{u}\hat{x}\right]\ket{n=0,t}=0
\end{align}

在坐标空间表述为:
\begin{align}
    \left\{u(t)\frac{\hbar}{i}\frac{\partial}{\partial x'}\right\}\braket{x'|n=0,t}=0
\end{align}

归一化后的解为:
\begin{align}
    \braket{x'|n=0,t}=\left(\frac{mv}{\pi\hbar}\right)^{1/4}\frac{1}{\{u(t)\}^{1/2}}\exp\left[\frac{i m}{2\hbar}\frac{\dot{u}(t)}{u(t)}x'^2\right]
\end{align}

与静态势谐振子完全类似,可以通过创建算子$\hat{C}_S^\dagger(t)$对基态重复操作来创建完全正交基的所有其它状态:
\begin{align}
    \ket{n,t}=\frac{\left[\hat{C_S^\dagger(t)}\right]^n}{\sqrt{n!}}\ket{n=0,t}\label{eq:basic_sattes}
\end{align}

将$u(t)$如公式\eqref{eq:ut_expression}重写后,在坐标空间表述为:
\begin{align}
    \braket{x'|n,t}=\exp\left[-i\left(n+\frac{1}{2}\right)\nu t\right]\chi_n(t) \label{eq:quantum_states_expression}
\end{align}

其中,$H_n$是$n$阶厄米多项式,$\chi_n(t)$表达式如下:
\begin{align}
    \chi_n(t)=\frac{1}{\sqrt{2^n n!}}\left(\frac{m\nu}{\pi  \hbar}\right)^{1/4}
    \frac{\exp\{-i n \arg\left[\Phi(t)\right]\}}{\{\Phi(t)\}^{1/2}}\\
    \times H_n\left\{\left[\frac{m\nu}{\hbar|\Phi|^2}\right]^{1/2}x'\right\}\\
    \times \exp\left\{\frac{m\nu }{2\hbar}\left[1-\frac{i\Phi(t)}{\nu\Phi(t)}\right]x'^2\right\}
\end{align}

经典的微运动作为射频驱动场周期的脉动出现在波函数中。对于静态势谐振子,能量本征态的演化只将波函数乘以相位因子(这就是为什么它们被称为静止态)。在此处研究的时间相关电势场中,同样如此,不同之处仅在于这里时间只能取得RF周期$T=2\pi/\omega_{rf}$的整数倍。公式\ref{eq:quantum_states_expression}给出的状态并不是能量本征态(它们周期性地与驱动场交换能量,类似于经典的微运动),但它们是时间相关势中可能的平稳状态的很好的近似。因此,它们通常被称为\emph{准平稳状态(Quasistationary States)}。

紧接着的小结节中将介绍与第\ref{section:lowest_order_approximation}节中提出的经典伪势解类似的量子力学中的运动解的低阶近似,找到对静态势谐振子图景的最低阶修正。


\subsubsection[量子低阶近似]{量子低阶近似}
量子力学中的低阶近似从导出$u(t)$的近似表达开始。与经典的情况类似,低阶近似需要满足条件:$|a_x|,q_x^2\ll 1$、$C_{\pm 4}=0$。结合公式\eqref{eq:boundary_condition}中的初始条件可以得到:
\begin{align}
    \beta_x\approx\sqrt{a_x+q_x^2/2},\ \nu\approx\beta_x\omega_{rf}/2,\\
    u(t)\approx\exp{i\nu t}\frac{1+(q_x/2)\cos(\omega_{rf}t)}{1+q_x/2}\label{eq:quantum_lowest_order_approximation}
\end{align}

这实质上就是前面第\ref{section:lowest_order_approximation}节在公式\eqref{eq:classical_motion_solution}中找到的经典解。
仍然必须强调的是,只有在这种低阶近似中,参考谐振子的频率$\nu$才等于特征指数$\beta_x\omega_{rf}/2$。现在很明显地可以看出$\chi_n(t)$以周期$T_{rf}=2\pi/\omega_{rf}$被调制。具体可以从基态波函数的近似表达式$\chi_0(t)$中看到:
\begin{align}
    \chi_0(t)=\left(\frac{m\nu}{\pi \hbar}\right)^{1/4}\sqrt{\frac{1+q_x/2}{1+(q_x/2)\cos(\omega_{rf}t)}}
    \times \exp\left(\left\{i\frac{m\omega_{rf}\sin(\omega_{rf}t)}{2\hbar\left[1/q_x+\cos(\omega_{rf}t)\right]}-\frac{m\nu}{2\hbar}\right\}x'^2\right)
\end{align}

而公式\eqref{eq:quantum_states_expression}中的相位因子由基态伪能量$\hbar\nu/2$控制。如果设置$\omega_{rf}=0$,则这个表达式与静态谐波势基态波函数相同。



\subsection[囚禁离子的光场耦合]{囚禁离子的光场耦合}
% \textcolor{red}{这部分主要参考文献\cite[p3-8]{Leibfried_Blatt_Monroe_Wineland_2003}}

想要用离子实现量子计算,需要用一定的方式作用于离子对其进行操控。通过对离子施加合适的电磁场,可以使得囚禁离子的内部能级相互相干耦合,并且也可以与离子的外部运动自由度耦合。
囚禁良好和耦合度高的离子-光子体系可以被看作为
\emph{Jaynes-Cummings 哈密顿量(Jaynes-Cummings Hamiltonian)}\cite[]{Janszky_Yushin_1986}
因此,许多致力于囚禁离子相干相互作用的工作都受到这种耦合在量子光学中所起的重要作用的启发。
除了上面这种特殊情况之外,其余多种可能的情况下会涉及到多个运动量子之间的相互交换,类似于量子光学中的多光子跃迁。
此外,离子-光子耦合中隐含的能量守恒不必局限在囚禁离子的内部态和运动态之间的转换,也可以实现囚禁离子内部态的相互转换,比如吸收来自耦合光场的能量跃迁到更高的能级上。
最后,如果考虑了运动方面的全量子力学图景,包括微运动引起的修正,则另一类跃迁可能需要被考虑,涉及到在离子阱的RF电势中运动态整数倍数的相互转换或驱动场整数倍数的组合和长期运动(微运动边带)。

% \subsection[囚禁离子的光场耦合]{囚禁离子的光场耦合}
\subsubsection[二能级近似]{二能级近似\label{section:two_level_approximation}}
在常规的离子阱研究中,会把囚禁离子的电子能级结构近似为\emph{二能级系统(Two-level System)},这为研究提供了很大的方便。这个二能级系统表示为$\ket{g}$和$\ket{e}$,他们之间有着$\hbar \omega=\hbar(\omega_e-\omega_g)$的能量差。这对于实际的囚禁离子不总是适用的,仅在光场与离子两能级近似共振耦合且耦合的拉比频率远强于衰减到其它态的强度时才成立。不过这个条件对当今研究的多数实验系统中的离子(如镱离子、钡离子、钙离子等等)来说都是成立的。
相应的二能级哈密顿量$\hat{H}^{(e)}$的表述如下:
\begin{align}
    \hat{H}^{(e)}=\hbar(\omega_g\ket{g}\bra{g}+\omega_e\ket{e}\bra{e})\\
    =\hbar\frac{\omega_e+\omega_g}{2}(\ket{g}\bra{g}+\ket{e}\bra{e})\\
    +\hbar\frac{\omega}{2}(\ket{g}\bra{g}-\ket{e}\bra{e}) \label{eq:two_level_hamiltonian}
\end{align}

任何二能级系统有关的算子都可以被映射到$1/2$自旋算子基矢上,因此上述的$\hat{H}^{(e)}$及其相关的算子也可以被表示为泡利矩阵的形式,它们之间的映射关系如下:
\begin{align}
    \ket{g}\bra{g}+\ket{e}\bra{e}\mapsto \hat{I},\ \ket{g}\bra{e}+\ket{e}\bra{g}\mapsto \hat{\sigma}_x,\ \\
    i(\ket{g}\bra{e}-\ket{e}\bra{g})\mapsto \hat{\sigma}_y,\ \ket{e}\bra{e}-\ket{g}\bra{g}\mapsto \hat{\sigma}_z
\end{align}

在这种映射情况下,$\hat{H}^{(e)}$可以被表述为:
\begin{align}
    \hat{H}^{(e)}=\hbar\frac{\omega}{2}\sigma_z
\end{align}

相应的能量以$-\hbar(\omega_e+\omega_g)/2$重新缩放,以抑制公式\eqref{eq:two_level_hamiltonian}中与状态无关的能量贡献。

\subsubsection[耦合的理论表述]{耦合的理论表述\label{section:coupling_theory}}

为了以一种简单而充分的方式描述囚禁离子与光场的相互作用,如前一节所述,我们假设囚禁离子的运动在所有三个维度上都是简谐的。
下面的描述将包括囚禁电势的显式时间依赖,但在许多情况下,将离子的运动建模为三维静态势谐振子是足够的。
因为如果无量纲Paul阱参数$a_x$和$q_x^2$的模量相对于与静态势和射频势(见第\ref{section:ion_classical_motion}节)远小于1,则一般理论只会引入非常微小的变化。这对于实验中常用的离子阱是成立的。
内部状态和运动耦合的广义描述遵循文献\cite[]{Cirac_Garay_Blatt_Parkins_Zoller_2002,1996Paul}中的方法。

另外还假定,在光场的多极展开中处理最低阶展开就足够了,在所讨论的近共振电子状态之间产生一个不退化的矩阵元。
电子波函数的展宽远小于耦合场的波长这一事实证明了这一假设是合理的。
对于偶极允许跃迁,将用偶极近似来处理场,而对于偶极禁止跃迁,只考虑场的四极分量。
对于拉曼跃迁,近共振的中间能级将被绝热消除,使这些跃迁在形式上等同于其它类型的跃迁。

% \subsubsubsection[总哈密顿量和相互作用哈密顿量]{总哈密顿量和相互作用哈密顿量\label{section:total_hamiltonian}}
系统的总哈密顿量可以写作如下形式:
\begin{align}
    \hat{H}=\hat{H}^{(m)}+\hat{H}^{(e)}+\hat{H}^{(i)}
\end{align}

其中$\hat{H}^{(m)}$是沿着离子阱轴向的运动哈密顿量,如在第\ref{section:quantum_motion}节公式\eqref{eq:static_harmiltonian_oscillator}中讨论过的;
$\hat{H}^{(e)}$代表如第\ref{section:two_level_approximation}节中所述的离子的内部电子能级结构;
$\hat{H}^{(i)}$代表本部分将要讨论施加的光场与离子之间的耦合。

电偶极跃迁、电四极跃迁和受激拉曼跃迁可以在一个统一的框架中描述,该框架将某个共振拉比频率$\Omega$、有效光频率$\omega$和有效的波矢量$\mathbf{k}$与这些跃迁类型中的每一种相关联。
% 电偶极跃迁和电四级跃迁的耦合光场的频率和波矢量是相同的,但两者驱动受激拉曼跃迁的光场频率差$\omega=\omega_1-\omega_2$,波矢量差$\mathbf{k}=\mathbf{k_1}-\mathbf{k_2}$。
% 电偶极跃迁和电四级跃迁的耦合光场驱动受激拉曼跃迁的光场频率相差$\omega=\omega_1-\omega_2$,波矢量差$\mathbf{k}=\mathbf{k_1}-\mathbf{k_2}$,而两者的频率和波矢量是相同的。
电偶极跃迁和电四级跃迁的耦合光场的波矢量和频率是相同的,而这两者与驱动受激拉曼跃迁的光场存在着$\omega=\omega_1-\omega_2$的频率差和$\mathbf{k}=\mathbf{k_1}-\mathbf{k_2}$波矢量差。

对于行波光场,可以用以下形式的耦合哈密顿量来描述上面所有三种跃迁类型:
\begin{align}
    \hat{H}^{(i)}=(\hbar/2)\Omega(\ket{g}\bra{e}+\ket{e}\bra{g})\\
    \times\left[e^{i(k\hat{x}_S-\omega t + \phi)}+e^{-i(k\hat{x}_S-\omega t + \phi)}\right]
\end{align}

在$1/2$自旋代数中我们可以将其重新表述为:
\begin{align}
    \ket{e}\bra{g}\mapsto\hat{\sigma}_+=1/2(\hat{\sigma}_x+i\hat{\sigma}_y),\\
    \ket{g}\bra{e}\mapsto\hat{\sigma}_+=1/2(\hat{\sigma}_x-i\hat{\sigma}_y)
\end{align}

为了便于说明和理解,我们以一个维度的阐述为例。有效波矢量$\mathbf{k}$选为沿着离子阱中的$x$轴方向。转换到相互作用表象,可以得到自由哈密顿量$\hat{H}_0=\hat{H}_{(m)}+\hat{H}_{(e)}$与相互作用哈密顿量$\hat{V}=\hat{H}_{(i)}$的最简单的一个动力学图景。记$\hat{U}_0=\exp[-(i/\hbar)\hat{H}_0t]$,转换后的相互作用哈密顿量为:

\begin{align}
    \hat{H}_{int} &=\hat{U}_0^\dagger\hat{H}^{(i)}\hat{U}_0\\
    &=(\hbar/2)\omega e^{(i/\hbar)\hat{H}^{(e)}t}(\sigma_++\sigma_-)\\
    &\times e^{-(i/\hbar)\hat{H}^{(e)}t}e^{(i/\hbar)\hat{H}^{(m)}t}\left[e^{i(k\hat{x}-\omega t + \phi)}+e^{-i(k\hat{x}-\omega t + \phi)}\right]e^{-(i/\hbar)\hat{H}^{(m)}t}\\
    &=(\hbar/2)\Omega(\sigma_+e^{i\omega_0 t}+\sigma_-e^{-i\omega_0 t}) e^{(i/\hbar)\hat{H}^{(m)}t} \\
    &\left[e^{i(k\hat{x}-\omega t + \phi)}+e^{-i(k\hat{x}-\omega t + \phi)}\right]e^{-(i/\hbar)\hat{H}^{(m)}t}
\end{align}

上面公式表述中与时间相关的振动项提取出来后就是$\exp[\pm i (\omega\pm \omega_0)t]$。这两项一项振动频率为$\delta_f=\omega+\omega_0$,是快速振荡项;另一项振动为$\delta=\omega-\omega_0\ll \omega_0$,是慢速振荡项。在研究中我们一般会忽略快速振动项的贡献,也就是所谓的\emph{旋波近似(Rotating-wave Approximation)}。

引入\emph{Lamb-Dick参数(Lamb-Dick Parameter, LDP)}$\eta=kx_0$,其中$x_0=\sqrt{\hbar/(2m\nu)}$是参考振荡子基态波函数的$x$轴方向的扩展,海森堡图景下的$\hat{x}(t)$表述为:
\begin{align}
    k\hat{x}(t)=\eta\left\{\hat{a}u^*(t)+\hat{a}^\dagger u(t)\right\}
\end{align}

相互作用哈密顿量经过旋转波近似后的最终形式为:
\begin{align}
    \hat{H}_{int}(t)=(\hbar/2)\Omega\hat{\sigma}_+ \exp(i\{\phi+\eta[\hat{a}u^*(t)+\hat{a}^\dagger u(t)]-\delta t\})+H.c.
\end{align}

指数项中的时间依赖性由频率差$\delta$和$u(t)$控制。
考虑公式\eqref{eq:ut_expression}中的解形式及Lamb-Dicke 参数下的拓展有:
\begin{align}
    &\exp(i\{\phi + \eta [\hat{a}u^*(t)+\hat{a}^\dagger u(t)]-\delta t\})\\
    &=e^{i(\phi-\delta t)}\sum_{m=0}^{\infty} \frac{(i\eta)^m}{m!}\left\{\hat{a}e^{-i\beta_x\omega_{rf}t} \sum_{n=-\infty}^{\infty}C_{2n}^* \times e^{-i n\omega_{rf}t+H.c.}\right\}^m
\end{align}

很容易验证任何时刻的衰减满足下式:
\begin{align}
    (l'+l\beta_x)\omega_{rf}=\delta
\end{align}

其中$l$和$l'$为整数且$l\neq l', if l'\neq 0$,它们是哈密顿量中的两项慢速变化,贡献了含时变化的主要部分(其余部分可以忽略)。如果其中一个调制边带与静止离子的跃迁频率$\omega_0$重合,则该边带可以诱导离子内部状态跃迁。实际上,由于实验中$(|a_x|,q_x^2)\ll 1$,因此公式\eqref{eq:quantum_lowest_order_approximation}中的$\beta_x\omega_{rf}\approx\nu$且$C_o\approx(1+q_x/2)^{-1}$。于是相互作用哈密顿量就可以简化成如下形式:
\begin{align}
    \hat{H}_{int}(t)=(\hbar/2)\Omega_0\sigma_+ \exp\{i\eta(\hat{a}e^{-i\nu t}+\hat{a}^\dagger e^{i\nu t})\}e^{i(\phi-\delta t)}+ H.c. \label{eq:interaction_hamiltonian}
\end{align}

其中缩放的相互作用强度为$\Omega_0=\Omega/(1+q_x/2)$,这个缩放反映了射频驱动频率下波包振荡引起的耦合减少。

% \textcolor{red}{这里的波包 ‘breathing’ 应该怎么翻译?}

% \subsubsubsection[拉比频率]{拉比频率\label{section:rabi_frequency}}

依赖失谐变量$\delta$,公式\eqref{eq:interaction_hamiltonian}中的哈密顿量会耦合一定的内态和运动态。如果公式\eqref{eq:interaction_hamiltonian}在$\eta$范围内,这会产生一个包含$\sigma_{\pm}$的组合项,它有$l$个$\hat{a}$算符和$m$个$\hat{a}^\dagger$算符并以频率$(l-m)\nu=s\nu$转动。
当$\delta\approx s\nu$时,这些组合将是共振的,同时会将多态$\ket{g}\ket{n}$和$\ket{e}\ket{n+s}$耦合起来。
% 对于$s>0(s<0)$的情况,耦合强度通常称为$|s|$级\emph{蓝(红)边带}拉比频率,
对于$s>0$的情况,耦合强度通常称为$|s|$级\emph{蓝边带}拉比频率,
对于$s<0$的情况,耦合强度通常称为$|s|$级\emph{红边带}拉比频率,
表达式为\cite[]{Leibfried_Meekhof_King_Monroe_Itano_Wineland_2002, Beige_Bose_Braun_Huelga_Knight_Plenio_Vedral_2000}:
\begin{align}
    \Omega_{n,n+s}=\Omega_{n+s,n}=\Omega_0|\braket{n+s|e^{i\eta(a+a^\dagger)}|n}|\\
    =\Omega_0 e^{-\eta^2/2}\eta^{|s|}\sqrt{\frac{n_<!}{n_>!}}L_{n_<}^{|s|}(\eta^2)
\end{align}

其中$n_<$是比$n+s$和$n$小,$n_>$是比$n+s$和$n$大,$L_n^\alpha(X)$是广义拉盖尔多项式:
\begin{align}
    L_n^\alpha(X)=\sum_{m=0}^{n}(-1)^m\begin{pmatrix}
        n+\alpha \\ n-m
    \end{pmatrix}\frac{X^m}{m!}
\end{align}

% \subsection[小结]{小结}

% 上述的两部分(第\ref{section:total_hamiltonian}和\ref{section:rabi_frequency}节)介绍了光与离子耦合的主要基础。实际上关于光与离子的耦合还有许多内容可以介绍,这些耦合特性也在离子量子计算中被使用到了,不过我并不打算在这里就将其一一说明。作为替代,我将在后续结合离子量子计算的相关操作来对这些特性做出相应的介绍,比如离子量子比特的冷却(第\ref{section:yb_laser_cooling}节)、初始化(第\ref{section:yb_state_init}节)、探测(第\ref{section:yb_state_detection}节)、操作(第\ref{section:yb_state_manipulation}节)等。











% ============================================================================
% ============================================================================
% =======================     镱离子量子计算    ===============================
% ============================================================================
% ============================================================================
\section[镱离子量子计算]{镱离子量子计算\label{section:yb_computation}}
目前所有实现的用于离子量子计算的离子都属于\emph{类氢离子(Hydrogen-like ions)}, 比如$\rm Be^+$(NIST\cite[]{Monroe_Meekhof_King_Itano_Wineland_2002,Lin_Gaebler_Reiter_Tan_Bowler_Wan_Keith_Knill_Glancy_Coakley_et_al_2016}),$\rm Mg^+$(NIST\cite[]{Barrett_Schaetz_DeMarco_Britton_Chiaverini_Itano_Jelenkovic_Jost_Langer_Leibfried_et_al_2003, Wan_Kienzler_Erickson_Mayer_Tan_Wu_Vasconcelos_Glancy_Knill_Wineland_et_al_2019}),$\rm Ca^+$(Univer-sity of Innsbruck\cite[]{Lanyon_Hempel_Nigg_Müller_Gerritsma_Zähringer_Schindler_Barreiro_Rambach_Kirchmair_et_al_2011,Monz_Nigg_Martinez_Brandl_Schindler_Rines_Wang_Chuang_Blatt_2016}, University of Oxford\cite[]{Ballance_Harty_Linke_Sepiol_Lucas_2016,Schäfer_Ballance_Thirumalai_Stephenson_Ballance_Steane_Lucas_2018}),$\rm Ba^+$(Washington University\cite[]{Dietrich_2009,Dietrich_Kurz_Noel_Shu_Blinov_2010},UCLA\cite[]{Hucul_Christensen_Hudson_Campbell_2017}),$\rm Yb^+$(University of Maryland\cite[]{Olmschenk_Younge_Moehring_Matsukevich_Maunz_Monroe_2007,Debnath_Linke_Figgatt_Landsman_Wright_Monroe_2016}, University of Sussex\cite[]{Weidt_Randall_Webster_Lake_Webb_Cohen_Navickas_Lekitsch_Retzker_Hensinger_2016})。
类氢离子具有最简单的能级结构,处理起来相对其它类型的离子要简单很多。它们具有的内态$^2S_{1/2}$到$^2P_{1/2}$之间的闭环光学跃迁,能方便地实现激光冷却、高保真初始化和内态读出操作\cite[]{Harty_Allcock_Ballance_Guidoni_Janacek_Linke_Stacey_Lucas_2014},这对于量子计算的实现至关重要。要挑选出合适的类氢离子,还有一些其它的参数需要被考虑,如离子的质量、循环跃迁的波长、核自旋的值和$^2D$能级的亚稳态的寿命等\cite[]{Bruzewicz_Chiaverini_McConnell_Sage_2019}。


要用离子实现量子计算,首先需要选择合适的离子体系,并且需要对选定离子的能级结构和量子态控制原理和方法有着充分的了解。出于历史和现实的原因,我们实验室选择$\rm ^{171}Yb^+$离子作为实现量子计算的物理平台。在接下的几节中,我将阐述用镱离子实现量子计算的一些重要的概念,如镱离子的能级结构和比特编码方式、镱离子的态初始化、镱离子的探测、镱离子的操控、基于离子的通用量子门构建等。


\subsection[镱离子的能级结构和和比特编码方式]{镱离子的能级结构和和比特编码方式}
在第\ref{section:introduction}章介绍中我们介绍过实现量子计算对物理平台的要求,最基本的要求是要能找到合适的量子系统来编码量子比特,即$\ket{0}$、$\ket{1}$。离子的内态是十分稳定并且易于操控的,因此一般选择离子的两个内部状态来编码量子信息并执行计算。对于我们现在系统中采用的$\rm ^{171}Yb^+$离子来说,用来编码量子比特的是有着$1/2$自旋的处于$^2S_{1/2}$基态的超精细能级\cite[]{Olmschenk_Younge_Moehring_Matsukevich_Maunz_Monroe_2007}。如图\ref{fig:energy_structure}所示,$\ket{0}$和$\ket{1}$分别被编码到了基态的相应状态,具体来说是分别是$\ket{0}=\ket{F=0,m_F=0}$和$\ket{1}=\ket{F=1,m_F=0}$,当外磁场为零时两个能级的能量差为$12.64$GHz。

\begin{figure}
    \centering
    \caption[镱离子的能级结构和比特编码方式]{镱离子的能级结构和比特编码方式\label{fig:energy_structure}}
    \includegraphics[width=1.0\linewidth]{energy_structure}
\end{figure}

通过在阱外向阱中离子施加的约$6$高斯的静磁场会使离子产生约$11$Hz的二阶\emph{Zeeman}能级劈裂。从$\ket{1}$态向$^2P_{1/2}$能级跃迁回路的驱动波长在$369.526$nm附近,这个波段的激光有成熟的商业激光系统供应,同时对光纤传输也很友好。略有遗憾的是这个$\ket{1}{^2P}_{1/2}$跃迁回路并不是完全闭合的,因为存在从$^2P_{1/2}$到$^2D_{3/2}$的微弱态泄露,分支比约为$0.5$\%\cite[]{Olmschenk_Younge_Moehring_Matsukevich_Maunz_Monroe_2007}。不过这个可以通过使用一个带$3.07$GHz边带的$935$nm波长的激光来将泄露的态再回泵浦到整个回路中。另外,背景撞击可能导致离子跑到$^2F_{7/2}$使离子“熄灭”,这个可以通过一个$638$nm波长的激光器来克服,重新将离子“点亮”。值得注意的是,这个任务除了$638$nm波长的激光器可以完成外,$760$nm\cite[]{Huntemann_Okhapkin_Lipphardt_Weyers_Tamm_Peik_2012}、$355$nm\cite[]{Senko_2014}也可以实现。另外,由于从$^2F_{7/2}$到$^3F[1/2]_{3/2}$的跃迁大部分共振在$375.856$nm波长上,因此$375$nm波长的激光也可以完成。

\subsection[镱离子的电离和囚禁]{镱离子的电离和囚禁}
为了能够研究离子的特性我们首先需要将原子电离并稳定地囚禁住。
离子的囚禁采用的是
% 如第\ref{section:ion_classical_motion}节中介绍的
动态囚禁的方式,离子的电离一般是通过激光电离的方式实现的。对于$\rm ^{171}Yb^+$离子来说,电离的方式是采用的是\emph{两步电离法(two-step ionization)}\cite[]{Olmschenk_Younge_Moehring_Matsukevich_Maunz_Monroe_2007},这种方法可以避免紫外激光的使用。
首先,我们使用$398.991$nm波长的激光将镱原子从$^1S_0$能级激发到$^1P_1$能级;然后再使用一个低于$394.088$nm波长的激光将最外层电子激发到连续态区域中去,比如通常采用的$369.526$nm波长激光,或者$355$nm波长、$375$nm波长。
在实践中,金属镱一般填充在原子炉中并被放置在真空室中。需要激发时就通过施加恒定电流加热原子炉,使原子从原子炉中热喷射出来的,其中会有一小部分穿过离子阱的中心。$398.911$nm和$369.525$nm波长的激光束在离子阱中心与原子炉中喷射出的原子重叠。于是,该区域的原子就可以被电离,并在高效冷却后被离子阱动态囚禁。由于同位素位移,我们可以通过微调$398.911$nm激光的波长来选择性地电离镱离子不同的同位素。为了抑制多普勒频移的影响,激光器的路径最好垂直于原子流。

\begin{figure}
    \centering
    \caption[刀片阱实物图]{实验中刀片阱实物图\label{fig:trap_blads}}
    \includegraphics[width=0.8\linewidth]{trap_blads.jpg}
\end{figure}

% 如第\ref{section:ion_classical_motion}节中给介绍的,
由\emph{Earnshaw's theorem}\cite[]{Earnshaw}我们知道离子没有办法被静态地囚禁在三维空间中,
因此实验中对离子采用动态方式进行囚禁,实际囚禁离子的离子阱如\ref{fig:trap_blads}图所示,这种类型的离子阱一般被称为刀片阱。它会产生如下形式的赝势$\Phi_p$:
\begin{align}
    \Phi_p=\frac{eV^2}{4MR^4\Omega_{rf}^2}\sum_{m}^{}\alpha_m^2r_m^2
\end{align}

其中$V$,$R$,$M$分别是RF电压、电极到阱中心的距离和离子的质量。$\Omega_{rf}$是RF的频率,$\alpha_m$是$r_m$方向的系数。
这个电势的结果是在所有方向上都有正的系数 ($\alpha_m^2>0$),直观地揭示了囚禁带电离子的能力。同时还可以从中得到有效的阱频率$\nu_m=eV\alpha_m/\sqrt{2}MR^2\Omega_{rf}$。

\subsection[镱离子的激光冷却]{镱离子的激光冷却\label{section:yb_laser_cooling}}
从原子炉中喷出的原子是十分“热”的(标准大气压下金属镱的熔点高达$824 ^o C$),由麦克斯韦运动分布可知它们的速度能达到超过一百米每秒。如此热的离子很容易从离子阱的囚禁电势中逃脱,很难被囚禁更不用说进行什么别的操作了。因此,我们需要某种冷却手段来给离子“降温”。

\begin{figure}
    \centering
    \caption[激光施加到离子的多普勒效应示意图]{激光施加到离子的多普勒效应示意图\label{fig:doppler_cooling_scattering}}
    \includegraphics[width=0.45\linewidth]{doppler_cooling_scattering}
\end{figure}
离子可以使用红色失谐的相对传播激光束进行激光冷却,这被称为多普勒冷却\cite[]{Hänsch_Schawlow_1975}。如图\ref{fig:doppler_cooling_scattering}所示,当离子与激光反向运动时,吸收一个光子离子将减少$\hbar \vec{k}$的动量;当离子与光同向运动时,吸收一个光子离子将增加$\hbar \vec{k}$的动量。随后离子将通过受激辐射的形式随机散射$\hbar \vec{k}$的动量,这个随机散射过程的平均结果是动量变化为$0$。

\begin{figure}
    \centering
    \caption[多普勒冷却示意图]{多普勒冷却示意图\label{fig:doppler_cooling}}
    \includegraphics[width=1.0\linewidth]{doppler_cooling}
\end{figure}
如图\ref{fig:doppler_cooling}所示,由于多普勒效应,在离子局部坐标系下看入射激光的波长会产生$-\vec{k}\cdot\vec{v}$的频移。因此如果施加红是失谐的激光,那么与激光方向相反的离子将比与激光方向相同的离子吸收更多的光子;另一方面,对打的激光会抵消激光本身对离子的加速效应,最终离子会失去较多的动量,被激光“冷却”下来。

对$\rm ^{171}Yb^+$离子来说,在这个过程中跃迁$^2S_{1/2}\ket{F=1}\to {^2P}_{1/2}\ket{F=0}$和$^2S_{1/2}\ket{F=0}\to {^2P}_{1/2}\ket{F=1}$同时被驱动以覆盖离子的所有超精细能级,避免离子陷入暗态(变黑)。
此外,冷却激光的所有偏振都施加上去来提高冷却效率,相关跃迁回路和离子散射图如图\ref{fig:doppler_cooling_level}所示。
\begin{figure}
    \centering
    \caption[离子冷却跃迁回路示意图]{离子冷却跃迁回路示意图\label{fig:doppler_cooling_level}}
    \includegraphics[width=0.6\linewidth]{doppler_cooling_level}
\end{figure}

\subsection[镱离子的态初始化]{镱离子的态初始化\label{section:yb_state_init}}

几乎任何量子计算或模拟任务都以确定性的纯状态开始。获得这个起点的过程称为\emph{态初始化(State Initialization)},这是通过囚禁离子平台中的光泵浦过程实现的。

\begin{figure}
    \centering
    \caption[离子态初始化示意图]{离子态初始化示意图\label{fig:initialization}}
    \includegraphics[width=0.6\linewidth]{initialization}
\end{figure}

如图\ref{fig:initialization}所示,在初始化过程中从$^2S_{1/2}\ket{F=1}$到$^2P_{1/2}\ket{F=1}$的跃迁被驱动。离子的状态能够退激发到$\ket{0}$状态,而由于$12.6$GHz的失谐,处于$\ket{0}$态的离子不会被进一步的激发。对于单个离子,这一过程在$5$μs内就可以以非常高的初始化保真度完成。这篇\cite[]{Harty_Allcock_Ballance_Guidoni_Janacek_Linke_Stacey_Lucas_2014}文献展示了状态初始化的误差小于可以$10^{−4}$。

\subsection[镱离子的态探测]{镱离子的态探测\label{section:yb_state_detection}}
在经过各种中间操控后,量子计算的结果需要被读出,这一过程通过离子的态测量实现。如图\ref{fig:detection}所示,通常采用状态相关的荧光检测技术进行高保真读出\cite[]{Blinov_Leibfried_Monroe_Wineland_2004}。在这个过程中使用的是$^2S_{1/2}\ket{F=0}$到$^2P_{1/2}\ket{F=1}$之间的跃迁($^2S_{1/2}\ket{F=0}$到$^2P_{1/2}\ket{F=0}$之间的跃迁是偶极禁止跃迁的)。因此,我们可以通过收集到的散射光子的数量的多少来区分投影状态。
\begin{figure}
    \centering
    \caption[离子态测量示意图]{离子态测量示意图\label{fig:detection}}
    \includegraphics[width=0.6\linewidth]{detection}
\end{figure}

态测量过程收集到的散射光子可以用电子倍增电荷耦合器件(Electron-multiplying charge-coupled device, EMCCD)或者光电倍增管(Photomultiplier Tube, PMT)通过两步成像来计数\cite[]{Wong_Campos_Johnson_Neyenhuis_Mizrahi_Monroe_2016}
% ,如图\ref{fig:two_step_imaging}所示
。
对于单个离子,为了量化量子比特状态检测的保真度,我们可以将量子比特准备为$\ket{0}$状态或$\ket{1}$状态,然后应用持续$40$μs的探测激光束同时计数收集到的光子。每个状态的序列通常重复$2000$次以获得统计分布。
以我们目前系统为例,入射光子数量的统计数据遵循泊松分布,$\ket{0}$和$\ket{1}$状态的平均光子分别为$0.05$和$9.58$。分离良好的分布使我们能够将投影状态与固定阈值区分开来。对于单次检测,如果收集到的光子数大于1,则我们将投影状态分类为$\ket{1}$状态,否则投影状态被视为$\ket{0}$状态。

% \begin{figure}
%     \centering
%     \caption[两步成像系统]{两步成像系统\label{fig:two_step_imaging}}
%     \includegraphics[width=0.6\linewidth]{two_step_imaging}
% \end{figure}



\subsection[镱离子的态操控]{镱离子的态操控\label{section:yb_state_manipulation}}
对离子量子比特的操控可以通过微波\cite[]{Olmschenk_Younge_Moehring_Matsukevich_Maunz_Monroe_2007}或脉冲激光\cite[]{Lee_2005}实现。微波对离子的态操控实现简单,但是由于所用微波的波长太长,使用微波技术方案往往仅限于单比特门;激光操控离子有诸多优点,尤其是在大规模的集成和寻址方面。接下来的小节将具体介绍这两类离子比特门操控方式。
\subsubsection[微波操控镱离子]{微波操控镱离子\label{section:microwave_operation}}
本小节将简要介绍微波操控的离子量子门,以帮助建立离子态操控的一般方式。
在微波场下的离子哈密顿量为(保持$\hbar=1$):
\begin{align}
    \hat{H}=\frac{\omega_q}{2}\hat{\sigma}_z + \Omega\cos\left(\vec{k}\cdot\vec{r}-\omega t + \phi\right)\hat{\sigma}_x
\end{align}

第一项是简化的离子比特二能级系统的静态能量,其能量差为$\omega_q$;第二项是磁偶极跃迁的导出项,其中磁场的振动频率为$\omega$,初始相位为$\phi$;$\Omega$是代表耦合强度的拉比频率;当$\omega\approx 12.6$GHz时有$\vec{k}\approx3\times ^{-4}um^{-1}$,所以空间项$\vec{k}\cdot \vec{r}$相对比较小,可以被忽略。通过一个重映射变换$\hat{H}_I=e^{iH_0t}(\hat{H}-\hat{H}_0)e^{-iH_0t}$,其中$\hat{H}_0=\omega\hat{\sigma}_z/2$,可以将自由空间哈密顿量转化到相互作用表象下:
\begin{align}
    \hat{H}_I&=e^{iH_0t}(\hat{H}-\hat{H}_0)e^{-iH_0t}\\
    &=-\frac{\mu}{2}\hat{\sigma}_z+\frac{\Omega}{2}\left(\hat{\sigma}_+e^{i\omega t}+\hat{\sigma}_-e^{-i\omega t}\right)\left(e^{i(\omega t-\phi)}+e^{-i(\omega t-\phi)}\right)\\
    &\approx -\frac{\mu}{2}\hat{\sigma}_z+\Omega(\hat{\sigma}_+e^{i\phi}+\hat{\sigma}_-e^{-i\phi})\\
    &=-\frac{\mu}{2}\hat{\sigma}_z+\frac{\Omega\cos{\phi}}{2}\hat{\sigma}_x+\frac{\Omega\sin{\phi}}{2}\hat{\sigma}_y\label{eq:interaction_hamiltonian_microwave}
\end{align}

其中$\mu=\omega=\omega_q$是量子比特和单微波光子之间的能量差。在$\approx$这步,我们采用了在第\ref{section:coupling_theory}节中介绍过的\emph{旋波近似},将含$2\omega$的快速振动项忽略掉了。相互作用表象的优点在于最终的哈密顿量是不含时的,这样各个泡利矩阵的成分的概率幅就完全由施加的微波场决定了。
对于一个给定的状态$\ket{\Psi(0)}=c_0(0)\ket{0}+c_1(0)\ket{1}$,其中$|c_0(0)|^2+|c_1(0)|^2=1$,它的时间演化由演化算符$U(t)$决定:
\begin{align}
    \ket{\Psi(t)}=U(t)\ket{\Psi(0)}=\exp(-iH_It)\ket{\Psi(0)}
\end{align}

如果给定$\ket{\Psi(0)}=\ket{0}$,那么可以解析地写出这个态的演化后的概率幅:
\begin{align}
    c_1(t)&=-i\frac{\Omega}{\Omega_{eff}}e^{i\phi}\sin{\frac{\Omega_{eff}t}{2}},\\
    c_0(t)&=\cos{\frac{\Omega_{eff}t}{2}}-i\frac{\mu}{\Omega_{eff}}\sin{\frac{\Omega_{eff}t}{2}}
\end{align}

其中$\Omega_{eff}=\sqrt{\Omega^2+\mu^2}$是有效拉比频率\cite[]{Foot_2005}。$\ket{1}$态的概率$p_1(t)=|c_1(t)|^2$在时间$t=\pi/\Omega_{eff}$时取得最大值,最大值为$\Omega^2/(\Omega^2+\mu^2)$。不同失谐$\mu$情况下的拉比振荡如图所示。

\begin{figure}
    \centering
    \caption[不同失谐$\mu$情况下的拉比振荡]{不同失谐$\mu$情况下的拉比振荡\cite[]{Lu_2019}}
    \includegraphics[width=1.0\linewidth]{rabi_oscillations}
\end{figure}

在量子计算的实现中,我们总是使$\mu\approx0$以实现任意单比特的旋转操控:
\begin{align}
    R_\phi(\theta)=U\left(\frac{\theta}{\Omega},P\phi\right)=\exp[-i\frac{\theta}{2}\sigma_\phi]
\end{align}

其中$\sigma_\phi=\sigma_x\cos{\phi}+\sigma_y\sin{\phi}$。对于初始状态,在$\theta=\pi$或者说是$t=\pi/\Omega$时,初始态的概率可以完全转移到二能级系统的另一个态上。这种$R_\phi(\theta)$旋转操作被称为沿着$\phi$轴的态旋转操控,$R_\phi(\pi)$被称为$\pi$翻转,它的作用类似经典的NOT门。除此之外,还有一种重要的旋转操作$R_\phi(\pi/2)$,它常被称为$\pi/2$翻转,可以用来制备十分有用的叠加态$(\ket{0}+e^{-i\phi}\ket{1})/\sqrt{2}$。



\subsubsection[连续激光操控镱离子]{连续激光操控镱离子\label{section:raman_transition}}

离子的内态也可以用激光通过称为\emph{受激拉曼跃迁(Stimulated Raman Transition, SRT)}的过程来操控。与微波场操控的方式相比,光学激光器的波长更短,提供了强大量子比特寻址的优势,这对于多量子比特情况下的单个量子比特操作至关重要,接下来的部分介绍了这种方法。


信息储存在镱离子的超精细能级上,$\ket{0}$和$\ket{1}$能级之间有着$12.6$GHz的能量差。尽管激光的能量远大于这个能级间隔以至于无法用激光直接操控$\ket{0}$态和$\ket{1}$态之间的跃迁,我们仍然可以通过两光子的受激拉曼跃迁来实现对离子内态的操控,如图\ref{fig:raman_transition}所示。
通过同时激发从$\ket{0}$和$\ket{1}$到一个中间激发态$\ket{e}$的电偶极跃迁,整个系统的自由空间哈密顿量可以写为:
\begin{align}
    \hat{H}&=\frac{\omega_q}{2}\hat{\sigma}_z+\left(\omega_e+\frac{\omega_q}{2}\right)\ket{e}\bra{e}\\
    &+\frac{\Omega_1}{2}(\ket{e}\bra{0}+\ket{e}\bra{1}+H.c.)\left(e^{i(\vec{k}_1\cdot\vec{r}-\omega_1t+\phi_1)}+H.c.\right)\\
    &+\frac{\Omega_2}{2}(\ket{e}\bra{0}+\ket{e}\bra{1}+H.c.)\left(e^{i(\vec{k}_2\cdot\vec{r}-\omega_2t+\phi_2)}+H.c.\right)
\end{align}

\begin{figure}
    \centering
    \caption[镱离子受激拉曼跃迁示意图]{镱离子受激拉曼跃迁示意图。$\ket{e}$:上级激发态能级;$\ket{0},\ket{1}$:编码量子信息的高低能级;$\omega_{q}$:编码量子信息的能级能隙;$\Delta$:激光相对于上级激发态能级的失谐;$\mu$:用于施加数态跃迁的失谐;$\omega_e$:$\ket{1}$和$\ket{e}$能级的能隙;$\omega_1,\phi_1,\vec{k}_1, \Omega_1$和$\omega_2,\phi_2,\vec{k}_2, \Omega_2$:施加到离子量子信息编码能级上的激光频率、相位、波矢、耦合强度等参数。\label{fig:raman_transition}}
    \includegraphics[width=0.6\linewidth]{raman_transition}
\end{figure}

其中$\Omega_i,\vec{k}_i, \phi_i$分别为第$i$个拉曼激光的电偶极跃迁强度、波矢和光相位。$H.c.$表示厄米共轭。如果在旋转坐标系$\hat{H}_0=\frac{\omega_1\hat{\sigma}_z}{2}+\left(\omega_e+\frac{\omega_q}{2}\right)\ket{e}\bra{e}$下看,相互作用的哈密顿量变为了:
\begin{align}
    \hat{H}=
    &\left(\frac{\Omega_1}{2}e^{i(-\Delta-\mu)t}e^{i(\vec{k}_1\cdot \vec{r}+\phi_1)}
    +\frac{\Omega_2}{2}e^{i(-\Delta+\omega_q)t}e^{i(\vec{k}_2\cdot \vec{r}+\phi_2)}\right)\ket{e}\bra{0}\\
    +&\left(\frac{\Omega_1}{2}e^{i(-\Delta-\mu-\omega_q)t}e^{i(\vec{k}_1\cdot \vec{r}+\phi_1)}
    +\frac{\Omega_2}{2}e^{i(-\Delta)t}e^{i(\vec{k}_2\cdot \vec{r}+\phi_2)}\right)\ket{e}\bra{1}+H.c.\label{eq:interaction_hamiltonian_yb}
\end{align}

其中失谐$-\Delta$如图\ref{fig:raman_transition}所示.考虑$\Delta\gg\Omega_1,\Omega_2$且$|\Delta|\gg\omega_q\gg\mu$的情况,这时中间激发态$\ket{e}$可以绝热消除,最终离子比特系统中有效的哈密顿量如下:
\begin{align}
    \hat{H}_{eff}=\frac{\delta_{diff}}{2}\hat{\sigma}_z+\frac{\Omega}{2}\left(\hat{\sigma}_+e^{i\left(\vec{k}\cdot\vec{r}-\mu t+\phi\right)}
    +\hat{\sigma}_-e^{-i\left(\vec{k}\cdot\vec{r}-\mu t+\phi\right)}\right)\label{eq:effective_hamiltonian_raman}
\end{align}

其中$\Omega=\frac{\Omega_1\Omega_2}{2\Delta}$,$\vec{k}=\vec{k}_1-\vec{k}_2$,$\phi=\phi_1-\phi_2$分别是有效拉比频率、有效波矢和激光的相对相位。$\delta_{diff}$有如下形式\cite[]{James_Jerke_2007}:
\begin{align}
    \delta_{diff}&=-\left(\frac{\Omega_1^2}{4(-\Delta-\mu-\omega_q)+\frac{\Omega_2^2}{4(-\Delta)}}\right)+\left(\frac{\Omega_1^2}{4(-\Delta+\mu)+\frac{\Omega_2^2}{4(-\Delta+\omega_q)}}\right)\\
    &\approx-\frac{\Omega_1^2+\Omega_2^2}{4\Delta}\frac{\omega_q}{\Delta}
\end{align}

这个是量子比特能级的差分光位移,由于在一般的实验设置中$\omega_a/|\Delta|\sim 10^{-3}\ll 1$,$\delta_{diff}$的值非常小。这项可以通过将原来的旋转坐标$\omega_q$重新选择为$(\omega_q+\delta_{diff})$来吸收掉。
另外注意,这里的有效哈密顿量与公式\eqref{eq:interaction_hamiltonian_microwave}的共振情况相同,后者是用微波进行的操控。因此如果我们让$\vec{k}_1=\vec{k}_2$且$\mu=\delta_{diff}\approx0$,那么这两种离子比特的操控方式从结果上看没有什么不同。不过实际上,两者还有所不同。除了上面提高过的受激发的拉曼跃迁方法有单比特寻址的能力外,它还通过改变传播拉曼光束的角度$\vec{k}\in[0,4\pi/\lambda]$(激光方向从同向到反向传播)来提供波矢$\vec{k}$的操控自由度。因为波长$\lambda$是在光学波段,最大的波矢模$|\vec{k}|$能达到$10um^{-1}$,这为离子的运动耦合提供了可能性。离子也可以看作是一个以频率$\nu$振动的量化的谐振子
% (第\ref{section:ion_classical_motion}节)
,这时公式\eqref{eq:effective_hamiltonian_raman}中的哈密顿量为:
\begin{align}
    H_{eff}=\nu a^\dagger a+\frac{\Omega}{2}\left(\sigma_+e^{i(\eta(a^\dagger a)-\mu t+\phi)}+H.c.\right) \label{eq:effective_hamiltonian_rotating}
\end{align}

其中$a^\dagger,a$分别是产生和湮灭算子。这里我们通过研究一个维度的情况来具体说明,比如$x$方向。此时$\vec{k}\cdot\vec{r}\to k\cdot x$等价于$k\Delta x(a+a^\dagger)=\eta(a+a^\dagger)$,其中$\Delta x=\sqrt{\hbar/2M\nu}$是基态波函数的展宽,$\eta=k\Delta x$是在第\ref{section:coupling_theory}节介绍过的LDP。将公式\eqref{eq:effective_hamiltonian_rotating}转换到$H_0=\nu a^\dagger a$的表象下并将$\exp{i\vec{k}\cdot \vec{r}}$进行一阶展开,可以得到近似的相互作用哈密顿量:
\begin{align}
    H_I\approx\frac{\Omega}{2}\left[\sigma_+(1+i\eta a e^{-i\nu t}+i\eta a^\dagger e^{i\nu t})e^{-i\mu t+i\phi}+H.c.\right]
\end{align}

这个近似形式仅在$\eta\sqrt{\braket{(a+a^\dagger)^2}}\ll1$时成立,这也被称为\emph{Lamb-Dick近似(Lamb-Dick Approximation)},其中$\braket{\cdot}$表示量子态平均。在实验中这个条件是成立的,不然就会产生高阶的跃迁。通过调节$\mu$等于$0,-\nu,\nu$,我们可以得到三种类型的跃迁:
\begin{align}
    H_{car}&=\frac{\Omega}{2}(\sigma_+e^{i\phi}+H.c.)\\
    H_{rsb}&=\frac{i\eta\Omega}{2}(\sigma_+ae^{i\phi}+H.c.)\\
    H_{bsb}&=\frac{i\eta\Omega}{2}(\sigma_+a^\dagger e^{i\phi}+H.c.)
\end{align}

$H_{car}$, $H_{rsb}$, $H_{bsb}$分别被称为载带跃迁、红边带跃迁和蓝边带跃迁,如图\ref{fig:red_carray_blue_sideband}所示。离子比特的两个内能级分别为$\ket{0},\ket{1}$;灰色先表示载带跃迁,其拉比频率为$\Omega$;红色线和蓝色线分别表示红边带和蓝边带跃迁,相应的拉比频率为$\eta \Omega$;$\nu$表示两个数态之间的能隙,也就是实现红蓝边带跃迁需调节的失谐量。

\begin{figure}
    \centering
    \caption[载带跃迁、红边带跃迁和蓝边带跃迁示意图]{载带跃迁、红边带跃迁和蓝边带跃迁示意图。$\ket{0},\ket{1}$:编码量子信息的高低能级;$\Omega$:驱动激光耦合强度;$\eta \Omega$:数态跃迁的耦合强度;$\nu$:数态之间的失谐能隙;$\ket{0}_F,\ket{1}_F,\ket{2}_F,\dots$:不同的数态的编号。\label{fig:red_carray_blue_sideband}}
    \includegraphics[width=0.8\linewidth]{red_carray_blue_sideband}
\end{figure}

载带跃迁可以用来实现如微波操控一样(第\ref{section:microwave_operation}节)的单比特门翻转$R_\phi(\theta)$。
红边带跃迁和蓝边带跃迁的哈密顿量分别称为\emph{Jaynes-Cummings}或\emph{anti-Jaynes-Cummings}模型\cite[]{Karnieli_Fan_2023}。这类跃迁可以将量子比特与其外部量子化的运动耦合,这对于实现多个离子量子比特纠缠至关重要。


\subsubsection[脉冲激光操控镱离子]{脉冲激光操控镱离子\label{section:pulsed_laser_ion_operation}}
如上一节所述的在用激光操控离子的技术方案下,$\ket{0}$和$\ket{1}$态耦合的主要方式是通过受激拉曼跃迁。脉冲激光控制离子的原理本质上与连续激光的情况没有不同。主要区别在于连续激光是通过驱动拉曼跃迁来控制的,而脉冲激光是通过同时驱动一系列小间隔的拉曼跃迁来控制的。原理图如图\ref{fig:pulsed_laser_interaction}所示。
\begin{figure}
    \centering
    \caption[脉冲激光操控镱离子]{脉冲激光操控镱离子。$\ket{0},\ket{1}$:编码量子信息的高低能级;$\omega_R$:激光重复频率间隙;$\omega_q$:编码量子信息的能级能隙;$\omega_0$:脉冲激光中心频率;$\Delta$:与上级激发态能级的失谐量。\label{fig:pulsed_laser_interaction}}
    \includegraphics[width=0.8\linewidth]{pulsed_laser_interaction}
\end{figure}

在空间中的固定点上,理想化的激光脉冲束有一个时间独立的电场$E(t)$:
\begin{align}
    E(t)=\sum_{n=1}^{N}f(t-nT)e^{i\omega_0t}
\end{align}

其中$f(t)$是脉冲激光的包络,$T$是激光脉冲的重复周期,$N
$是整数激光包含的脉冲数,$\omega_0$是脉冲激光的波长。脉冲激光在频域中可以表示为:
\begin{align}
    E(\omega)=\sum_{n=0}^{\infty}\tilde{f}(\omega)(\omega_0\pm n\omega_R)
\end{align}

其中$\tilde{f}(\omega)=\mathcal{F}(f(t))$表示$f(t)$傅里叶变换的结果,$\omega_R=2\pi\nu_R$是脉冲激光的重复频率。如图\ref{fig:pulsed_laser_comb}所示,脉冲激光在频域上是一系列频率梳齿,离子比特态之间拉曼跃迁的共振拉比频率由梳齿的所有谱分量的总和给出,如图\ref{fig:pulsed_laser_interaction}所示。在绝热近似消除激发态$^2P_{1/2}$并执行旋波近似后,脉冲激光的受激拉曼跃迁拉比频率有以下形式\cite[]{Hayes_Matsukevich_Maunz_Hucul_Quraishi_Olmschenk_Campbell_Mizrahi_Senko_Monroe_2010}:
\begin{align}
    \Omega=\frac{|\mu|^2\sum_{l}^{}E_lE_{l-q}}{\Delta}\approx\Omega_0\left(\frac{\omega_q\tau}{e^{\frac{\omega_q\tau}{2}}-e^{-\frac{\omega_q\tau}{2}}}\right)
\end{align}

\begin{figure}
    \centering
    \caption[脉冲激光在频域中的梳齿]{脉冲激光在频域中的梳齿\cite[Chap1.3]{Mizrahi_2013}\label{fig:pulsed_laser_comb}}
    \includegraphics[width=1.0\linewidth]{pulsed_laser_comb}
\end{figure}
% 基态和激发态之间的偶极矩阵元素
其中$\mu$是基态和激发态之间的偶极矩阵元素,$\tau\approx 1 $ps是每个脉冲的持续时间,$E_k\equiv \nu_R\tilde{f}(2\pi k\nu_R)$是激发态电子态,$q\equiv\omega_q/\omega_R$是整数。在上述的近似表述中,求和被表述为积分,每个脉冲表示为$f(t)=\sqrt{\pi/2}E_0 sech(\pi t/\tau)$($\tau\ll T$)。$\Omega_0=\nu_R\tau|\nu E_0|^2/\Delta=s\gamma^2/2\Delta$脉冲束的时间平均谐振拉比频率,$\gamma/2\pi=20$MHz是$^2P_{1/2}$态的辐射线宽,$s=\bar{I}/T_{sat}$是$^2S_{1/2}\leftrightarrow ^2P_{1/2}$的饱和强度。在假设单脉冲频谱展宽与我们的实验系统中满足的超精细频率($\omega_q\tau\ll 1$)相比较大的情况下,可以得到$\Omega\approx\Omega_0$。








% \section[Mølmer-Sørensen 门]{Mølmer-Sørensen 门}
\section[基于离子的通用量子门]{基于离子的通用量子门}

通用量子门是构建量子计算机的基本单元,它可以通过组合和嵌套来实现各种量子计算操作,如量子比特的翻转、受控幺正操作等,想要实现离子量子计算就需要适用于离子量子比特的通用量子门。两比特纠缠门可以通过利用不同离子之间共享的集体运动实现。该策略利用了囚禁离子链的一个非常有利的特性,称为\emph{完全连通性(Full Connectivity)},这意味着整个离子链的离子都有相互作用。

Ignacio Cirac和Peter Zoller最早在1995年中提出了双量子比特纠缠门的第一个方案\cite[]{Cirac_Zoller_2002},称为\emph{Cirac-Zoller门(Cirac-Zoller Gate, CZ-Gate)},并在2003年在实验中实现\cite[]{Schmidt_Kaler_Häffner_Riebe_Gulde_Lancaster_Deuschle_Becher_Roos_Eschner_Blatt_2003}。
然而,如果离子不冷却到它们的运动基态,CZ-Gate的性能会急剧下降。这种苛刻的要求限制了CZ-Gate的扩展。
Klaus Mølmer和Anders Sørensen在1999年提出了另一种运动状态不敏感门的技术方案\cite[]{Sørensen_Mølmer_2002},被称为\emph{Mølmer-Sørensen门(Mølmer-Sørensen Gate, MS-Gate)}。
由于MS-Gate的更鲁棒的特征,它广泛用于囚禁离子量子计算中。在本节将详细介绍MS型双量子比特纠缠门的实现。

如果我们对离子施加二色场$\mu_b=\mu, \mu_r=-\mu$,就可以得到自旋耦合力,相应的哈密顿量可以表述为:
\begin{align}
    H_{SDF}=\sum_{j=1}^{2}\sum_{m=1}^{2}\frac{\eta_{j,m}\Omega}{2}\left(a_m^\dagger e^{i\delta_m t}e^{-i\phi_p}+a_m e^{-i\delta_m t}e^{i\phi_p}\right)\hat{\sigma}_{\phi_s}^j
\end{align}

其中$\eta_{j, m}$是缩放的Lamb-Dicke参数(其中包括来自模式变换矩阵的因子)\cite[]{James_1998},$\hat{\sigma}_j$是第$j$个离子比特的泡利矩阵。
$a_m$和$a_m^\dagger$是第$m$个模式的湮灭和产生算子,相应的模式频率表示为$\nu_m$。$\delta_m=\nu_m-\mu$是相对于第$m$个模式的失谐,它对于红蓝两个相应的跃迁的拉比频率相等为$\Omega_r=\Omega_b=\Omega$。
光学相位$\phi_b$和$\phi_r$可以分解为运动和量子比特两部分,其形式分别为$\phi_p=(\phi_r-\phi_b)/2, \phi_s=(\phi_r+\phi_b+\pi)/2$。我们可以定义形式为$f_{j,m}=\eta_{j,m}\Omega e^{i\phi_p}/2$的一个复合力。不失一般性,我们可以设定$\phi_p=\phi_s=0$,从而得到一个由$\hat{\sigma}_x$决定的力,哈密顿量为:
\begin{align}
    H_{SDF}=\sum_{j=1}^{2}\sum_{m=1}^{2}\frac{\eta_{j, m}\Omega}{2}\left(a_m^\dagger e^{i\delta_mt}+a_m e^{-i\delta_mt}\right)\hat{\sigma}_x^j\label{eq:sigma_x_dependent_force}
\end{align}

那么由公式\eqref{eq:sigma_x_dependent_force}所描述的哈密顿量控制的演化可以推导出为:
\begin{align}
    U(\tau)=\mathscr{T}\exp\left[-i\int_{0}^{\tau}H_{SDF}(t)dt\right]\label{eq:unitary_evolution_operator}
\end{align}

这是一个时间顺序的积分,因为$[H_{SDF}(t_2),H_{SDF}(t_1)]\neq 0, t_1\neq t_2$。幸运的是,通过利用Magnus公式,方程\eqref{eq:unitary_evolution_operator}可以解析计算如下\cite[]{Lee_Brickman_Deslauriers_Haljan_Duan_Monroe_2005}:
\begin{align}
    U(\tau)=\exp\left\{-i\int_{0}^{\tau}H_{SDF}(t)dt-\frac{1}{2}\int_{0}^{\tau}dt_2\int_{0}^{t_2}[H_{SDF}(t_2),H_{SDF}(t_1)]dt_1\right\}\label{eq:hamiltonian_with_magnus}
\end{align}

具体来看,公式\eqref{eq:hamiltonian_with_magnus}中第一个积分变为:
\begin{align}
    T_1=&-i\int_{0}^{\tau}H_{SDF}(t)dt\\
    =&-i\int_{0}^{\tau}\sum_{j=1}^{2}\sum_{m=1}^{2}\frac{\eta_{j,m}\Omega}{2}\left(a_m^\dagger e^{i\delta_mt}+a_m e^{-i\delta_mt}\right)\hat{\sigma}_x^j dt\\
    =&\sum_{j=1}^{2}\sum_{m=1}^{2}\left(\alpha_{j,m}(\tau)a_m^\dagger-\alpha_{j,m}^*(\tau)a_m\right)\hat{\sigma}_x^j\label{eq:integral_t1}
\end{align}

其中,
\begin{align}
    \alpha_{j,m}(\tau)=-i\frac{\eta_{j,m}\Omega}{2}\int_{0}^{\tau}e^{i\sigma_m t}dt=\frac{\eta_{j,m}\Omega}{2}\frac{1-e^{i\delta_m\tau}}{\delta_m}\label{eq:alpha_j_m}
\end{align}

显然,第一个积分的结果揭示了在多个运动模式上的状态相关位移算子,它将量子比特状态与其在位置动量相空间中的位移纠缠在一起。
公式\eqref{eq:hamiltonian_with_magnus}中第二个双重积分可以简化为:
\begin{align}
    T_2=&-\frac{1}{2}\int_{0}^{\tau}dt_2\int_{0}^{t_2}[H_{SDF}(t_2),H_{SDF}(t_1)]dt_1\\
    =&-\sum_{j,j'=1}^{2}\sum_{m,m'=1}^{2}\frac{\eta_{j,m}\eta_{j',m'}\Omega^2}{8}\hat{\sigma}_x^j\hat{\sigma}_x^{j'}\\
    &\int_{0}^{\tau}dt_2\int_{0}^{t_2}dt_1\left[a_m^\dagger e^{i\delta_m t_2}+a_m e^{i\delta_m t_2}, a_{m'}^\dagger e^{i\delta_{m'} t_1}+a_{m'} e^{i\delta_{m'} t_1}\right]\\
    =&i\sum_{j,j'=1}^{2}\sum_{m,m'=1}^{2}\frac{\eta_{j,m}\eta_{j',m'}\Omega^2}{4}\hat{\sigma}_x^j\hat{\sigma}_x^{j'}\int_{-}^{\tau}dt_2\int_{0}^{t_2}dt_1\sin[\delta_m(t_2-t_1)]\\
    =&i\sum_{j,j'=1}^{2}\sum_{m,m'=1}^{2}\frac{\eta_{j,m}\eta_{j',m'}\Omega^2}{4}\left(\frac{\tau}{\delta_m}-\frac{\sin(\delta_m\tau)}{\delta_m^2}\right)\hat{\sigma}_x^j\hat{\sigma}_x^{j'}\\
    =&-i\theta_{1,2}(\tau)\hat{\sigma}_x^1\hat{\sigma}_x^2\label{eq:integral_t2}
\end{align}

其中,
\begin{align}
    \theta_{1,2}(\tau)=-2\sum_{m=1}^{2}\frac{\eta_{1,m}\eta_{2,m}\Omega^2}{4}\left(\frac{\tau}{\delta_m}-\frac{\sin(\delta_m\tau)}{\delta_m^2}\right)
\end{align}

由于$\hat{\sigma}_x^j\hat{\sigma}_x^j=\mathbb{I}$,所有自交互项都消失了,仅导致可忽略的恒定相位。耦合强度$\theta_{1,2}$中的两个因子是由于$\hat{\sigma}_x^1\hat{\sigma}_x^2$和$\hat{\sigma}_x^2\hat{\sigma}_x^1$的对称贡献产生的。
这里的量子比特-量子比特耦合项提供了纠缠不同量子比特的可能性。这里注意,由于$[H_{SDF}(t_3),[H_{SDF}(t_2),H_{SDF}(t_1)]]=0$,即使是高阶积分也会消失。

通过将公式\eqref{eq:integral_t1}和公式\eqref{eq:integral_t1}带入公式\eqref{eq:hamiltonian_with_magnus}中,我们可以得到简化的演化算子:
\begin{align}
    U(\tau)=\exp\left[\sum_{j=1}^{2}\sum_{m=1}^{2}\left(\alpha_{j,m}(\tau)a_m^\dagger-\alpha_{j,m}^*(\tau)a_m\right)\hat{\sigma}_x^j-i\theta_{1,2}(\tau)\hat{\sigma}_x^1\hat{\sigma}_x^2\right]
\end{align}

为了具有纯量子比特耦合,量子比特运动应该在时间$\tau$处解纠缠,这表明对于任何$j,m$来说$\alpha_{j,m}(\tau)$应该为零。由公式\eqref{eq:alpha_j_m}可知,这些等价于$\delta_m\tau$对任意$m$来说都是$2\pi$的整数倍。
对于两量子比特链来说,这可以通过设置$\mu=(\nu_1+\nu_2)/2$和$\tau=2n\pi\times2/(\nu_1-\nu_2)$(假设$\nu_1>\nu_2$),其中$n$是正整数。这样一来,耦合力变为:
\begin{align}
    \theta_{1,2}=-\frac{n\pi\Omega^2}{\delta_1}\left(\frac{\eta_{1,1}\eta_{2,1}}{\delta_1}+\frac{\eta_{1,2}\eta_{2,2}}{\delta_2}\right)\approx-\frac{2n\pi\eta_{1,1}^2\Omega^2}{\delta_1^2}
\end{align}

最后的简化是由于$\delta_1=-\delta_2$和$\eta_{1,1}=\eta_{2,1}\approx\eta_{1,2}=-\eta_{2,2}$的关系。最终的演化算子揭示了双量子比特纠缠操作,可以写为:
\begin{align}
    XX_{1,2}(\theta_{1,2})=U(\tau)=\exp[-i\theta_{1,2}\hat{\sigma}_x^1\hat{\sigma}_x^2]
\end{align}

这是\emph{镜像型门(Ising-Type Gate)}之一,表示为XX门。通过改变驱动激光$\Omega$的功率来调节耦合强度。
XX门最重要的功能之一是制备纠缠态,如下公式所描述的:
\begin{align}
    XX(\theta)\ket{00}=\cos\theta\ket{00}-i\sin\theta\ket{11}
\end{align}

在选择$\theta$为$\pm\pi/4$时,制备出的状态是最大纠缠态$\ket{00}\pm i\ket{11}/\sqrt{2}$,相当于贝尔态。通常用这些状态的状态保真度来表征XX门的性能。

在量子计算领域,CNOT门是最著名的通用双量子比特门。XX门可以通过夹在几个单比特门之间转换为CNOT门\cite[]{Debnath_Linke_Figgatt_Landsman_Wright_Monroe_2016},如图\ref{fig:xx_gate}所示。
从这个意义上说,我们总是选择任意单量子比特旋转和双量子比特XX门,而不是CNOT门,作为我们的囚禁离子平台的通用门集。


\begin{figure}
    \centering
    \caption[XX门转换为CNOT门]{XX门转换为CNOT门\label{fig:xx_gate}}
    \includegraphics[width=1.0\linewidth]{xx_gate}
\end{figure}

\newpage
\section[章末小结]{章末小结}

相对于传统测控应用领域来说,量子物理实验系统对测控系统时间控制的精度和分辨率、处理器的计算能力、大规模的可拓展性等方面有着更高的要求。为了更好地认识和处理量子物理实验尤其是基于离子阱的量子计算实验系统的需求,本章介绍了面向离子阱量子计算的测控系统研究的重要知识背景。

离子的囚禁是进行离子阱量子计算的基本前提,有了可以在空间中稳定囚禁的离子才能进一步地利用其具备的量子特性进行量子计算的研究。在离子阱量子计算领域中囚禁离子最常使用的是Paul阱,它使用刀片电极生成射频电场对离子进行动态囚禁。本章介绍了此场景下离子的囚禁及其涉及到的经典和量子运动,同时介绍了离子与光场的耦合作为离子比特的态操控的理论基础。此外,对于我们实验室目前所使用的镱离子,系统介绍了它的能级结构、比特编码、激光冷却、量子态的初始化以及态探测等方面的内容,还包括基于微波、连续激光、脉冲激光的离子量子态操控方式和用于离子量子比特的通用量子门。

通过离子阱量子计算系统这些背景的理解,可以更加深刻地认识到量子系统对测控的实际需求。首先,量子比特操控的操控时间一般在毫秒到微秒级的,其对实时性的要求基本在百微秒至纳秒级,因此要求测控系统应该有微秒或纳秒级的任务时间精确控制能力;而且由于量子的概率特性,一般的实验操作如态的测量往往需要重复几百上千次来获得统计的结果,还有量子纠错等算法的实现需要系统具备实时在线的信息处理和决策能力,这对量子测控系统的实时计算能力有着很强的要求;另外真正实现通用量子计算机至少需要上千个全可控的量子比特,因此大规模的比特控制需要未来的量子测控系统具备强大的可拓展性并且能够用于构建相关系统助力提高量子比特操作的保真度。








