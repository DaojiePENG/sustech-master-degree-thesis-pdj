% !TeX root = ../sustechthesis-example.tex

% 中英文摘要和关键字

\begin{abstract}
  离子阱是当前最具发展前景的量子计算平台之一,它的实现需要电子学、光学、真空、测控等多种学科领域的支持。
  % 其中测控系统处于十分核心的地位,它将离子量子计算的各个部分联系起来,对量子比特进行调控并采集结果进行分析和处理,以实现量子计算的各种操作。
  随着量子计算技术的不断发展,传统测控系统以及早些年开发的量子测控系统逐渐难以适应日益增高的量子测控需求,为应对这些新挑战,本文面向离子阱量子计算系统:
  \begin{enumerate}
    \item 引入和分析了RTMQ测控系统架构。基于囚禁离子量子比特制备和控制的理论背景分析了量子物理实验对测控系统的实时性、可拓展性、集成性等方面的切实需求;针对这些需求介绍了一种面向离子量子计算系统的强实时、分布式、易拓展的RTMQ测控系统架构,以及与之相配套的汇编指令集与多节点间实时通信链路系统;
    \item 对螺线管谐振腔进行了系统研究。使用HFSS对谐振腔进行了有限元仿真并验证了仿真-实验结果一致性,分析了4项几何结构参数对其性能的影响并阐述了其耦合过程的特性,建立了谐振腔的数学模型并总结了设计特定频率下的高Q谐振腔的方法,给出了1种用于离子阱系统的易装配耦合的谐振腔优化设计。
    \item 设计制作了1套RTMQ测控板硬件及6个重要固件。基于FPGA设计并制作了1套适用于32位RTMQ系统的测控板硬件,单板集成了2路32位DDS、2路16位ADC、2路16位DAC、并提供64路GPIO和32对CPCI LVDS用户接口。用Verilog语言设计了UART、SPI、AD9910、ADC等4个芯片管理固件以及RTMQ系统的2个重要功能拓展固件——32位高速通用数字PID和32位高速通用数字IIR滤波器,板上工作频率分别可达200MHz和25MHz。
    \item 搭建了3个离子阱关键子的系统对这套RTMQ测控板卡进行了测试。使用这套RTMQ测控系统搭建和测试了离子阱频率稳定、激光幅度稳定和激光拍频稳定等3个离子阱关键子系统。将相关噪声对单比特$\theta=\pi$翻转操作保真度的影响分别由稳定前的$7.994\times10^{-6}$、$1.385\times 10^{-4}$、$4.992\times 10^{-1}$,降低到了稳定后的$3.734\times10^{-7}$、$1.646\times 10^{-6}$、$1.440\times 10^{-4}$,验证了RTMQ量子测控系统的有效性和优越性。
  \end{enumerate}
  
  % 以上研究工作满足了离子量子计算的切际需求,展示了所构建的RTMQ量子测控系统的有效性和优越性,促进了离子量子计算系统的数字化、集成化进程,为未来大规模通用量子计算做了测控方面的重要准备。

  % 关键词用“英文逗号”分隔,输出时会自动处理为正确的分隔符
  \thusetup{
    keywords = {离子阱量子计算, 谐振腔, 测控系统, FPGA},
  }
\end{abstract}

\begin{abstract*}
  Ion trap is a promising platform for Quantum-Computing(QC), which requires support of measurement and control system(MCS). 
  While traditional MCSs are becoming unfit for the increasing demand of QC. To address these new challenges, this article: 
  \begin{enumerate}
    \item Analyzes the Real Time Microsystem for Quantum physics (RTMQ) architecture. Based on the theoretical background of ion-QC, the practical requirements of QC on the real-time, scalability, integration and other aspects of the MCS are analyzed; According to these requirements, a strong real-time, distributed and easy to expand RTMQ MCS architecture for ion-QC system is introduced, as well as the assembly instruction set and multi node real-time communication link system;
    \item The helical resonator is studied systematically. The simulation of the resonator is carried out using HFSS, and the consistency of the sim-exp results is verified. The influence of four geometric parameters on its performance is analyzed, and the features of its coupling process are described. The mathematical model of the resonator is established, and the method of designing a high-Q resonator at a specific frequency is summarized. An optimized design of the resonator for easy assembly and coupling for ion-QC is given.
    \item A set of RTMQ board hardware and six firmware are designed and manufactured. Board suitable for 32-bit RTMQ system is designed based on FPGA. The board integrates 2 32-bit DDS, 2 16-bit ADC, 2 16-bit DAC, and provides 64 GPIO and 32 pairs of CPCI LVDS. 4 chip management firmware (UART, SPI, AD9910, ADC) and 2 important function expansion firmware (32-bit high-speed universal digital PID and 32-bit high-speed universal digital IIR filter) of RTMQ system are designed with Verilog. The on-board operating frequencies can reach 200MHz and 25MHz respectively.
    \item 3 ion-QC key subsystems were built to test the RTMQ board. Using this RTMQ board, three subsystems of ion-QC, including frequency stability, laser amplitude stability and laser beatnote frequency stability, were built and tested. The influence of correlated noise on the fidelity of single bit $\theta=\pi$ flip operation was reduced from $7.994\times10 ^ {-6}$, $1.385\times 10 ^ {-4}$, $4.992\times 10 ^ {-1}$ before stabilization to $3.734\times10 ^ {-7}$, $1.646\times 10 ^ {-6}$, $1.440\times 10^ {-4}$ verifying the merits of RTMQ board.
  \end{enumerate}
  
  % The above research meets the practical needs of ion-QC, demonstrates the effectiveness and superiority of the constructed RTMQ, promotes the digitalization and integration process of future large-scale general-purpose ion-QC systems in terms of MCS.

  % Use comma as seperator when inputting
  \thusetup{
    keywords* = { Ion-QC, Helical Resonator, Measurement and Control System, FPGA},
  }
\end{abstract*}
