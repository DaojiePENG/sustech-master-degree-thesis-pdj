% !TeX root = ../sustechthesis-example.tex

% 中英文摘要和关键字

\begin{abstract}
  量子计算因其强大的计算能力和潜在的应用前景而受到广泛关注。在量子算法的配合下,量子计算机可以实现许多通过经典计算机难以实现的计算,如大素数分解、量子多体系统仿真等。

  离子量子计算是当前最具发展前景的量子计算平台之一,它的实现需要电子学、光学、真空、测控等多种学科领域的支持。其中量子测控系统处于十分核心的地位,它将系统涉及的电子学、光学、真空等其余各个部分联系了起来,给出特定的微波信号、激光信号等策略对量子比特进行调控并采集结果进行分析和处理,以实现量子计算的各种操作。

  本文面向离子量子计算系统介绍了囚禁离子量子比特的准备和控制和一种强实时性、分布式、易拓展的RTMQ量子测控系统架构以及与之相配套的指令集与时序控制结构,
  并给出若干重要片上外设拓展的设计及FPGA实现,如、、、高速通用数字PID及高速通用数字IIR滤波器等;
  另外,使用有限元仿真及实验的方法对离子阱系统中一种重要的测控器件——螺线管谐振腔进行了仿真、实验、建模、设计等方面的优化研究,给出了设计特定频率下的高Q谐振腔的方法和更方便易用的机械结构设计;
  在上述基础之上,给出了基于RTMQ系统的测控板硬件设计及一些重要的功能外设固件设计和基于FPGA的实现。并且结合实际的实验系统需求实现了离子量子计算场景下基于RTMQ系统的离子阱频率稳定、激光功率稳定和激光拍频稳定等几个关键子系统的搭建和测试,展示了RTMQ量子测控系统的优越性,极大地促进了离子量子计算系统的数字化、集成化进程,为未来大规模通用量子计算奠定了测控方面的重要基础。
  
  % 本文重点研究了电学系统中谐振腔,创新性地开发和引入了一种可支持分布式计算的用于量子物理实验的实时微系统(RTMQ),并且基于测控系统和光电系统对用脉冲激光实现量子门的关键子系统进行了搭建和原理验证。
  % 以使用脉冲激光实现离子量子门为目标,在电学方面对微波系统中的关键器件——谐振腔进行了系统的仿真和实验研究;在测控系统方面协助开发了RTMQ系统并将其应用到了脉冲光离子量子比特门的几个关键的子系统,实现了阱频率稳定、脉冲激光拍频锁定、激光功率稳定等系统的原理和实现,用RTMQ系统及相应外设对这些系统进行了集成化、数字化的升级。

  % 关键词用“英文逗号”分隔,输出时会自动处理为正确的分隔符
  \thusetup{
    keywords = {量子计算, 离子阱, 测控系统, FPGA},
  }
\end{abstract}

\begin{abstract*}
  Quantum computing has attracted extensive attention due to its potentially powerful computing capabilities. With the cooperation of quantum algorithm, quantum computer can realize many calculations that are difficult to be realized by classical computers like factorization of large prime numbers, simulation of quantum many body systems. 

  Ion quantum computing is one of the most promising quantum computing platforms at present. Its implementation needs the support of electronics, optics, vacuum, measurement and control and other fields. Among them, the quantum measurement and control system is in a core position. It connects the other parts of the system, such as electronics, optics, vacuum, and so on. It gives specific strategies such as microwave signal and laser signal to regulate and control the quantum bits and collect the results for analysis and processing, so as to realize various operations of quantum computing.

In this paper, a RTMQ quantum measurement and control system architecture with strong real-time, distributed and easy to expand is proposed for ion quantum computing system. The design and FPGA implementation of high-speed universal digital PID and high-speed universal digital filter are given;
In addition, the simulation, experiment, modeling and design of helical resonator, an important measurement and control device in the ion trap system, are optimized by using the finite element simulation and experiment method. The design method of high-Q resonator at a specific frequency and the more convenient mechanical structure design are given;
Based on the above, combined with the actual experimental system requirements, several key subsystems such as ion trap frequency stability, laser power stability and laser beatnote stability based on RTMQ system in the ion quantum computing scenario were built and tested, which showed the advantages of RTMQ quantum measurement and control system, greatly promoted the digitization and integration process of ion quantum computing system, and laid an important foundation for the measurement and control of large-scale universal quantum computing in the future.


  % Use comma as seperator when inputting
  \thusetup{
    keywords* = {Quantum Computating, Ion Trap, Measurement and Control System, FPGA},
  }
\end{abstract*}
