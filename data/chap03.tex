% !TeX root = ../sustechthesis-example.tex

\chapter[用于离子阱量子计算的RTMQ测控系统]{用于离子阱量子计算的RTMQ测控系统\label{section:fpga_rtmq}}

% \textcolor{red}{
% 这部分参考RTMQ的相关专利和文档介绍整个测控系统的情况... 
% }

离子量子计算系统的实现涉及到一些物理量的精确调控和测量,这既包括量上的精确性,也包括时间上的精确性。因此系统对测量和控制性能提出了很多新的要求,其中十分关键的一点就是对测控系统实时性的要求。为此我们提出了一套强实时、可拓展、分布式的用于量子物理实验的实时微系统(Real Time Microsystem for Quantum physics, RTMQ),作为离子量子计算的测控系统。在接下来的小节里我将介绍现有实时系统的优缺点以及RTMQ系统的设计架构。随后介绍RTMQ系统的两个重要片上拓展——高速通用数字PID和高速通用数字滤波器的设计实现,以此构建实现离子量子计算的若干重要子系统的数字化推进。

\section[现有实时测控系统]{现有实时测控系统}

% 现有的实时系统一般使用主频在数百MHz至GHz量级的通用微处理器或微控制器作为控制的主体,以计时器中断和时间片分配等方式实现实时控制。这一方案成立的前提在于,所需的时间控制精度与指令执行频率之间有3-6个数量级的差异,因而通用处理器架构中存在的一些诸如分支预判、乱序执行等导致指令执行顺序不确定的因素以及中断系统中存在的现场保护、控制权交接等额外开销导致的时间控制不确定性可以忽略不计。

现有实时系统以通用微处理器或微控制器为主体,通过计时器中断和时间片分配实现实时控制,前提是时间控制精度与指令执行频率差异大,可忽略处理器架构和中断系统的不确定性。然而近来随着量子技术的发展,量子物理实验系统也开始产生对数据处理、复杂流程控制和实时控制的需求。不同于传统行业,量子物理实验系统对时间控制的精度和分辨率的要求在纳秒量级、延迟要求在百纳秒至数十微秒量级,与当前微处理器的主频相当,从而前述的现有的实时控制方案难以满足需求\cite[]{junhua03}。

因此早年在量子物理实验领域内,通常用FPGA(现场可编程门阵列)设计特定的时序脉冲发生器来产生高时间精度的脉冲序列,以此作为其它实验设备的触发信号,进行准确的时序控制。然而,这种方案的灵活性较差,只能产生预定的序列,无法在实验中对实验数据进行即时的处理,或根据实验的中间结果对后续的流程进行及时的调整\cite[]{junhua01}。
% 近年来随着量子算法的发展,实验方案越来越复杂,实验流程中开始包含快速反馈的结构,即在实验过程中对实验目标进行测量,获得一些中间结果,而后对中间结果进行计算和处理,并进而确定后续的实验流程。中间结果的处理和后续流程的确定,一般要求在数十纳秒至数十微秒量级的时间内完成,并且执行时刻必须要严格确定。这要求实验的测控系统具有通用计算的能力,简单的时序脉冲发生器无法满足这一要求。
随着量子算法发展,实验方案复杂,需在数十纳秒至数十微秒内处理中间结果并确定后续流程,简单的时序脉冲发生器已无法满足,实验测控系统需具备通用计算能力。

% 当前领域内针对此问题的主要解决思路为,另置一与时序脉冲发生器紧密连接的通用微处理器,用来对实验数据进行即时处理和产生时序脉冲发生器的后续输出时序。这一方案能较好的满足系统规模较小且实验时序不太复杂的情形下的实时控制需求。然而,这一方案的问题之一在于,微处理器和时序脉冲发生器依然是相互独立的两个个体,而微处理器的执行时序有其内在不确定性;二者之间要保持同步,或者需要频繁地相互交换触发信号,或者需要在时序设计上预留出充足的余量以覆盖此不确定性的最坏情形,总之都会复杂化时序的设计并产生时间浪费。
当前解决问题的思路是另置通用微处理器处理实验数据和产生时序。但该方案存在微处理器和时序脉冲发生器相互独立、同步困难的问题,会复杂化时序设计并产生时间浪费。该方案的另一问题是,系统大时需多个时序脉冲发生器和微处理器,会导致拥塞和同步性问题,而主流微处理器的架构和协议难以实现精确同步。

% 这一方案的另一问题在于,当系统规模较大,一个时序脉冲发生器无法控制整个系统时,就需要同时使用多个时序脉冲发生器,而一个微处理器同时处理过多的实验数据、同时控制过多的时序脉冲发生器,将不可避免的产生拥塞,这会进一步加剧前述的同步性问题。而如果同时使用多个微处理器,则不同微处理器之间的同步性又将成为问题;当前主流的微处理器架构和指令集都是针对通用计算而优化的,主流的微处理器使用的通信协议都是针对高吞吐率而优化的,二者都难以实现精确的时序同步。









% ============================================================================
% ============================================================================
% =======================  RTMQ实时量子测控系统 ===============================
% ============================================================================
% ============================================================================
\section[RTMQ实时量子测控系统]{RTMQ实时量子测控系统}

相对于传统测控应用领域,对于离子阱量子计算研究来说,一种实时性更强、拓展性更好、更灵活的测控系统十分重要。为了满足离子量子计算当前以及未来的测控需求,我们提出了一种实时性拓展性好、具有分布式计算能力的量子测控系统架构——RTMQ(用于量子物理实验的实时微系统,Real Time Microsystem for Quantum physics)。RTMQ系统架构提供了一种新的量子物理实验平台实时测控系统架构,解决了上述的不足。在RTMQ系统架构中,通用计算和时序控制由同一微处理器实现,因此避免了两个独立的模块之间同步性的问题;同时树状结构的系统中每个节点都具有通用计算的能力,因此可以实现计算任务的分布式处理,避免了拥塞的问题\cite[]{junhua01}。



\subsection[RTMQ实时量子测控系统架构]{RTMQ实时量子测控系统架构}

\begin{figure}
    \centering
    \caption[RTMQ实时量子测控系统架构示意图]{RTMQ实时量子测控系统架构示意图\label{fig:rtmq_nodes_and_leaves_structure}}
    \includegraphics[width=0.6\linewidth]{rtmq/rtmq_nodes_and_leaves_structure}
\end{figure}

RTMQ架构主要用于基于FPGA或ASIC的兼具通用计算和高精度时序控制能力的微系统。系统的整体结构为树状结构,如图\ref{fig:rtmq_nodes_and_leaves_structure}所示,系统包含一个根节点,多个中间结点和多个叶节点;根节点通过网络、USB等方式与控制计算机相连。不同节点可位于同一PCB上,亦可位于不同PCB上。不同节点的时钟通过根节点进行对齐,各个节点都具有独立的控制输出和通用运算能力,能够分布式地完成实验的控制和信息处理。

\begin{figure}
    \centering
    \caption[RTMQ实时量子测控系统架构节点示意图]{RTMQ实时量子测控系统架构节点示意图\label{fig:rtmq_board_overal_structure}}
    \includegraphics[width=0.4\linewidth]{rtmq/rtmq_board_overal_structure}
\end{figure}

各个节点都由测控硬件板卡构成,一般而言一个板卡具有如图\ref{fig:rtmq_board_overal_structure}所示的结构,板卡上的FPGA或ASIC包含一个RTMQ节点,RTMQ节点通过控制FPGA或ASIC的输入输出与数模/模数转换等各类功能芯片进行交互以实现所需功能,同时通过实时通信链路与其上级和下级节点连接。



\subsection[RTMQ实时量子测控系统架构的节点内部模块]{RTMQ实时量子测控系统架构的节点内部模块}
\begin{figure}
    \centering
    \caption[RTMQ实时量子测控系统架构节点内部模块示意图]{RTMQ实时量子测控系统架构节点内部模块示意图\label{fig:rtmq_board_inner_structure}}
    \includegraphics[width=1.0\linewidth]{rtmq/rtmq_board_inner_structure}
\end{figure}

一个RTMQ节点的内部模块如图\ref{fig:rtmq_board_inner_structure}所示,包含一个32位的微处理器、一个寄存器文件、一系列外设模块和一个链路管理模块。其中微处理器包含流控制器、计时器、异常管理模块、触发管理模块和算术逻辑单元5个子模块;寄存器文件包含多个寄存器;外设可分为系统外设和功能外设,系统外设包括指令缓存、数据缓存、节点信息只读存储器以及地址栈和数据栈,功能外设用于实现具体的逻辑或时序功能,可包含多个。

RTMQ架构中包含的微处理器可受指令控制进入挂起状态,而挂起状态可受计时器或触发管理模块的控制恢复正常运行,如此,微处理器的指令流便可以按一定的时间间隔对齐或与外部信号对齐。同时,节点中的系统外设和功能外设的行为受关联寄存器的读写控制,即微处理器的指令与系统各模块的功能和时序有严格的对应关系。因此,本发明提供的架构可实现实时控制与通用计算在指令流层面的结合。
而配置指令插入中断的机制确保了节点对其下级节点的绝对控制,即使下级节点的微处理器处于挂起状态,依然不受影响。配置指令插入中断配合具有确定通信延迟的实时通信链路系统,即可实现时序确定的跨节点的即时反馈控制。

此外,RTMQ架构中每个节点都具有通用计算和时序控制能力,如此,大多数通用计算和时序生成都可以在叶节点或较近的中间结点完成,对于大规模系统不存在拥塞的问题,具有良好的可扩展性。
% \section[测控硬件组成]{测控硬件组成}
% \textcolor{red}{
% 1. 展示实物板卡照片;}
RTMQ测控板实物图如图\ref{fig:rtmq_board_real}所示。RTMQ体系的构成在上述整体硬件核心架构的基础上,还需要结合节点实时通信的链路系统\cite[]{junhua02}来完成节点内部和外部通信的同步,以及相应的实时控制和通用计算的指令集\cite[]{junhua03}来完成对系统的管理和控制。值得关注的是,从它的指令集中可以看到RTMQ测控系统系统是支持板上运算和信息处理的,这是与其它依靠时序发生器构成的测控系统本质不同的地方和优势所在。这些处理包括逻辑运算和算术运算,构成了一个微处理器。而RTMQ系统的微处理器从架构上与CPU有着本质的不同,虽然牺牲了些运算效率,RTMQ的数据处理是严格实时的。这种处理器架构同样也具有拓展性和灵活性,我们可以方便地为它拓展和定义多种运算方式,也可以为其添加外设模块等,以适应更多更复杂任务的需求。在下面几小节中,我将介绍一些基本运算及外设模块的实现和使用。

\begin{figure}
    \centering
    \caption[RTMQ测控板去壳单板实物图]{RTMQ测控板去壳单板实物图\label{fig:rtmq_board_real}}
    \includegraphics[width=1.0\linewidth]{rtmq/rtmq_board_real}
\end{figure}



% ============================================================================
% ============================================================================
% =======================      高速数字PID      ===============================
% ============================================================================
% ============================================================================

\section[基于FPGA的高速通用数字PID]{基于FPGA的高速通用数字PID\label{section:digital_pid}}
% \textcolor{red}{
% 1. 介绍数字PID功能、逻辑图、Vivado中的实现、优势;}
当前量子计算系统中大量采用PID控制器进行诸如激光功率稳定、激光相位稳定、激光拍频锁定等等控制系统的构建,常用的器件有如LB1005模拟伺服控制器。这类模拟控制器的确定在于稳定性较差且价格昂贵,此外它的集成度也比较差,往往需要占用较大的实验平台面积。除此之外这类模拟PID控制器往往采用机械旋钮进行参数调节,不利于进行实验系统的集成化和自动化。与之相比较,使用融合了高速通用数字PID的RTMQ测控系统可以同时完成对实验序列的操控和对设备关键参数的伺服控制。除此之外,采用也对系统完成了集成化和数字化,提高了系统的稳定性和易用性。

\subsection[数字PID]{数字PID}
数字PID控制器是一种常用的自动控制算法,用于实现对系统的闭环控制。PID控制器通过对系统的误差进行比例(Proportional)、积分(Integral)和微分(Derivative)计算,生成控制信号来调整系统的输出,以达到期望的控制效果。在量子测控系统中很多地方都需要用到闭环控制,比如激光的功率稳定、激光的波长稳定、离子阱频率稳定等。相对于模拟PID控制器,数字PID控制器具有结构简单、易于实现、控制灵活、工作稳定等优点,是RTMQ系统中的重要组成部分。

PID 控制器的数学表达式可以表示为:
\begin{align}
    u(t)= K_p e(t) + K_i \int_{0}^{t} e(\tau) d\tau + K_d \frac{d e(t)}{dt}
\end{align}

其中,$u(t)$是控制器的输出,$e(t)$是系统的误差,$K_p$、$K_i$和$K_d$分别是比例系数、积分系数和微分系数。
 
PID控制器的实现可以分为模拟PID和数字PID两种方式。模拟PID是通过模拟电路实现的,而数字PID是通过数字计算实现的。数字PID控制器通常使用微处理器或计算机来实现,其基本结构包括采样、计算和输出三个部分。数字 PID 控制器的实现步骤如下:
\begin{itemize}
    \item 采样:对系统的输入和输出进行采样,获取当前时刻的误差值e(t);
    \item 计算:根据采样得到的误差值,按照 PID 控制器的数学表达式计算控制信号u(t);
    \item 输出:将计算得到的控制信号输出到执行机构,调整系统的输出;
\end{itemize}

在数字PID控制器的实现中,需要对积分和微分操作进行离散化处理。常用的离散化方法有矩形法和梯形法。矩形法将积分区间划分为若干个相等的子区间,每个子区间的积分值近似为矩形的面积;梯形法将积分区间划分为若干个相等的子区间,每个子区间的积分值近似为梯形的面积。这一步骤在嵌入式系统中通常使用模拟数字转换(ADC)芯片来完成。

\subsection[数字PID的增量表达式]{数字PID的增量表达式}
硬件资源在FPGA中是十分宝贵的资源,而乘法器这样的器件往往会占用大量的硬件资源。因此在满足时序要求的情况下我们希望能够尽量减少乘法器的使用,PID的增量表达式就是一种可以有效减少PID硬件电路实现过程中乘法器需求个数的方式,下面将介绍这种方法。

\begin{figure}
    \centering
    \caption[数字PID结构示意图]{数字PID结构示意图。MULT:乘法器;Reg:寄存器;ADD:加法器;SUB:减法器;$y_t, u_t$:输入和输出信号;$ref$:参考值输入;$k_0, k_1, k_2$:增量表达下的PID控制参数。\label{fig:digital_pid_structure_16bits_s}}
    \includegraphics[width=1.0\linewidth]{rtmq/digital_pid_structure_16bits_s}
\end{figure}

离散化后的PID表达式为:
\begin{align}
    u(n)=k_p e(n)+k_i\sum_{j=-}^{n}e(j)+k_d[e(n)-e(n-1)]\\
    u(n-1)=k_p e(n-1)+k_i \sum_{j=0}^{n-1}e(j)+k_d [e(n-1)-e(n-2)]
\end{align}

由上面两式可以导出:

\begin{align}
    \Delta u(n)=&u(n)-u(n-1)\\
    =&k_0 e(n)+k_1 e(n-1)+k_2 e(n-2)
\end{align}

其中$k_0=k_p+k_i+k_d,\ k_1=-k_p-2k_d,\ k_2=k_d$,这个式子被称为PID的增量算法。采用这种形式的好处是避免了计算PID原始表达式中的无限积分项。在这种增量式的方式下,PID的控制输出可以表达为:
\begin{align}
    u(n)=u(n-1)+\Delta u(n)=u(n-1)+k_0 e(n)+k_1 e(n-1)+k_2 e(n-2)\label{eq:increment_pid}
\end{align}

按照上述式\eqref{eq:increment_pid}表示的增量式算法,数字PID实现的结构示意图如图\ref{fig:digital_pid_structure_16bits_s}所示。接口主要有参考$ref$、参数$k_0, k_1, k_2$等可配置输入,反馈信号$y_t$等系统回路输入,以及$u_t$等系统控制输出。用到的器件包括加法器/减法器(图示中红、橙色模块)、乘法器(图示中蓝色模块)、寄存器(图示中靛色模块)。

\subsection[高速通用数字PID的FPGA实现]{高速通用数字PID的FPGA实现}

% 增量式PID最终在FPGA中实现的结构图如图\ref{fig:digital_pid_structure_16bits}所示。

根据图\ref{fig:digital_pid_structure_16bits_s}所示的结构,可以在FPGA中实现硬件的PID。其中主要涉及到的数字加法器、数字乘法器,分别采用的是超前进位加法器和Booth乘法器。对于高速时序电路,尤其是要满足与RTMQ的实时性匹配等相关需求,在开发过程中需要注意流水线的设置和对齐,最终的实现硬件框图如图\ref{fig:digital_pid_structure_16bits}所示,其硬件输出测试和仿真对比如\ref{fig:pid_compare}图所示。从图\ref{fig:pid_compare}中可见,在Vivado中实现的高速PID硬件电路在$k_i=1, k_d=1$的情况下经过大概1000ns后即可以将系统的输出稳定在参考值1000附近,与直接使用MATLAB进行数字PID仿真输出结果一致。

\begin{figure}
    \centering
    \caption[16位数字PID的FPGA实现结构图]{16位数字PID的FPGA实现结构图上\label{fig:digital_pid_structure_16bits}}
    \includegraphics[width=1.0\linewidth]{rtmq/digital_pid_structure_16bits}
\end{figure}

\begin{figure}
    \centering
    \caption[16位数字PID的FPGA实现结果与仿真结果对比]{16位数字PID的FPGA实现结果与仿真结果对比。初始系统输出和调节输出都为0,参考数值为1000,$k_i=1, k_d=1$,Vivado综合输出时长为3500ns,仿真持续时间为1000$T_c$(系统仿真迭代次数)。\label{fig:pid_compare}}
    \includegraphics[width=1.0\linewidth]{rtmq/pid_compare}
\end{figure}

整个PID控制器使用了一个减法器、三个加法器、三个乘法器和若干寄存器。图\ref{fig:digital_pid_structure_16bits}是一个16位输入的数字PID,它的输出是32位的,该实例模块总共包含9级流水线,工作频率可达200MHz以上。对更低或者更高位数的数字PID控制器,可以通过替换相应位数的加减运算和乘法模块,并调整相应的寄存器位宽来方便地得到。







% ============================================================================
% ============================================================================
% =======================     高速数字滤波器    ===============================
% ============================================================================
% ============================================================================
\section[基于FPGA的高速通用数字滤波器]{基于FPGA的高速通用数字滤波器\label{section:digital_iir}}

在量子计算的实现过程中,对于噪声的抗争从未停止。而对抗噪声的一种十分有效的方式就是使用各种各样的滤波器来对信号进行过滤以提高有用信号的质量。本质来讲,滤波器是一种选频装置,可以使信号中特定的频率成分通过,而极大地衰减其它频率成分。利用滤波器的这种选频作用,可以滤除干扰噪声或进行频谱分析。
当前实验系统中广泛出现滤波器为模拟滤波器,但是随着量子测控系统的数字化集成化进行逐步加深,在很多内部信号的处理上也有相当多的滤波需求,比如消除ADC芯片采样的噪声等。我们特别地在RTMQ系统板卡上实现了相匹配的高速通用数字滤波器,来进一步实现对量子物理实验系统信号的控制和优化。


% \subsection[数字滤波器]{数字滤波器}
% \textcolor{red}{
% 1. 介绍数字滤波器功能、种类、基本原理,比如有限冲激响应滤波器、无限冲激响应滤波器等等;}


\subsection[无限冲激响应滤波器IIR]{无限冲激响应滤波器IIR}
数字滤波器的基本原理是利用离散系统的特性对输入信号进行加工和处理,改变输入信号的频率特性,从而达到选频、提高信噪比、消除干扰等目的。数字滤波器一般由延迟单元、加法器、乘法器、寄存器等基本运算单元组成。不同类型的数字滤波器有不同的实现方法和应用领域。在通信、语音处理、图像处理、信号处理等领域,数字滤波器都有着广泛的应用。整个离子阱系统中很多模块都有滤波器的需求以获取更高质量的信号来控制量子比特。为此,整个RTMQ系统中需要实现数字滤波器以供系统设计和实现使用。

有限冲击响应滤波器(FIR滤波器)的冲激响应在有限时间内衰减为零,其输出仅取决于当前和过去的输入信号值。FIR滤波器在保证幅度特性的同时,很容易做到严格的线性相位特性。无限冲击响应滤波器(IIR滤波器)的冲激响应理论上应会无限持续,其输出不仅取决于当前和过去的输入信号值,也取决于过去的信号输出值。

有限冲击响应滤波器(FIR滤波器)的优点是具有线性相位、稳定性好、容易实现等,缺点是阶数较高时,运算量较大,对存储空间要求较高。无限冲击响应滤波器(IIR滤波器)的优点是阶数较低时,运算量较小,对存储空间要求较低,缺点是相位非线性、稳定性较差、设计复杂等。实际滤波器设计经验表明,实现相同形状的滤波效果,IIR滤波器所需的计算资源远小于FIR滤波器。鉴于硬件实现是对资源十分敏感的,因此我们选用IIR滤波器来进行实现。

IIR滤波器的冲激响应理论上应会无限持续,其输出不仅取决于当前和过去的输入信号值,也取决于过去的信号输出值,用差分方程来表示一个滤波器,其表达式为:
\begin{align}
    y(n)=\sum_{k=1}^Na_ky(n-k)+\sum_{k=0}^Nb_kx(n-k)\label{eq:iir_filter}
\end{align}

 % \textcolor{red}{
% 2. 介绍数字通用滤波器功能、逻辑图、Vivado中的实现、优势;}

其中,$y(n)$表示输出信号,$x(n)$表示输入信号,$a_k$和$b_k$表示滤波器系数。本质上来说,IIR滤波器就是将输入和过去的输出按照某种方式加权计算获得最终输出结果的。对于一个给定阶数和数字位数的IIR滤波器,它的形状完全取决于系数$a_k$和$b_k$。因此,为了维持通用性,在硬件实现的过程中,我们应该把系数($a_k, b_k$)设计为可外部软件配置的。
以四阶为例,通用四阶IIR滤波器的结构框图如图\ref{fig:iir_filter_s}所示,图中$a_0-a_3, b_0-b_3$为IIR滤波器的可配置系数,主要用到的模块有寄存器、乘法器、加法器等。
\begin{figure}
    \centering
    \caption[IIR滤波器结构框图]{IIR滤波器结构框图\label{fig:iir_filter_s}}
    \includegraphics[width=1.0\linewidth]{rtmq/iir_filter_s}
\end{figure}


\subsection[高速通用数字IIR滤波器的FPGA实现]{数字IIR滤波器的FPGA实现}

根据图\ref{fig:iir_filter_s}给出的结构图和公式\eqref{eq:iir_filter},可以将其在FPGA中进行实现和验证。数字IIR滤波器在FPGA中的实现结果的结构图如图\ref{fig:iir_filter_vivado}所示,整体结构中超如了若干的流水线用于对齐$x(n)$和$y(n)$的时序。除此之外,流水线也被用于满足电路的时延要求。需要注意的是,由于上一时刻的输出结果$y(n-1)$需要参与到下一时刻的输出结果的计算中,因此从计算结果到迭代反馈之间仅能有一级流水线存在,否则输出结果就会出错。也即从乘法器输入到乘法器输出,以及加法器输入到加法器输出并反馈到乘法器输入这个回路中只能存在一级流水线。正因此,这个IIR滤波器无法工作在较高的频率上,需要在板上额外配置工作频率,该32位通用IIR滤波器设计在FPGA板上的工作频率为25MHz。

\begin{figure}
    \centering
    \caption[IIR滤波器实现结果的结构图]{IIR滤波器实现结果的结构图(Vivado)\label{fig:iir_filter_vivado}}
    \includegraphics[width=1.0\linewidth]{rtmq/iir_filter_vivado}
\end{figure}




\subsection[滤波器形状测量]{滤波器形状测量}
数字滤波器的形状设计可以采用MATLAB中的滤波器设计器来完成。在量子计算的试验系统中常常要应对的是高频信号噪声,低通滤波器是十分常见的需求,因此下面以低通滤波器为例进行说明。如图\ref{fig:filter_design_real1},通过MATLAB设计了一个截止频率在100kHz附近的巴特沃斯型迭代滤波器,结果为一个3阶的迭代滤波器,参数为:$a_0=16777216;
a_1=-33470572;
a_2=16693565;
a_3=0;
b_0=52;
b_1=105;
b_2=52;
b_3=0$。可以看到设计的理论截止频率(100kHz)和数字滤波器实际的截止频率(13.17kHz)有较大的差异。其原因主要有两方面:1. 数字滤波器的设计本身实际上是对模拟滤波器的近似,因而结果不绝对准确;2. 实际在FPGA中的硬件滤波器运算位宽有限,爹地啊过程中存在截断效应;3. 对数据的表征不够精确,测试的过程中用到模数转换(16位)等中间数字过程。

\begin{figure}
    \centering
    \caption[IIR滤波器设计仿真和实际测试结果]{IIR滤波器设计仿真和实际测试结果\label{fig:filter_design_real1}}
    \includegraphics[width=1.0\linewidth]{rtmq/filter_design_real1}
\end{figure}

在理论结果绘制中,对滤波器的结果进行类似FPGA硬件中的人工截断,得到的模拟结果如图\ref{fig:filter_design_real2}所示。可以看到此时理论设计的截止频率结果与实际硬件数字PID的截止频率结果变得更加接近了。尽管仍然存在一定的差异,已经可以给出比较有意义的指导信息了。整体上来看,硬件实现的数字滤波器在配置为低通滤波器的时候带宽会被压窄。因此如果需要100kHz的带宽,则需要在设计时适当放大这个带宽需求。最终实际的滤波器带宽可以通过测量来进一步验证是否符合使用要求,如果不符合则可以重新设计和测试。值得一提的是,在实现了通用的硬件IIR滤波的基础上,重新设计和配置$a_0-a_3, b_0-b_3$等参数是十分方便快捷的,仅需更改可配置寄存器的值即可。

\begin{figure}
    \centering
    \caption[IIR滤波器考虑截断的仿真和实际测试结果]{IIR滤波器考虑截断的仿真和实际测试结果\label{fig:filter_design_real2}}
    \includegraphics[width=1.0\linewidth]{rtmq/filter_design_real2}
\end{figure}

% IIR滤波器形状测量如图\ref{fig:iir_filter_shape}所示。
% \begin{figure}
%     \centering
%     \caption[IIR滤波器设计仿真和不同参数下测试结果]{IIR滤波器设计仿真和不同参数下测试结果\label{fig:iir_filter_shape}}
%     \includegraphics[width=1.0\linewidth]{rtmq/iir_filter_shape}
% \end{figure}